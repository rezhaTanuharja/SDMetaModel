% -----------------------------------------------------------------------------
% Master thesis in the study program computational mechanics
%
% B.Sc. Rezha Adrian Tanuharja - 03751261
% M.Sc. Felix Schneider (supervisor)
%
% chapters/introduction/background.tex
% Last edited 03 November 2023
% -----------------------------------------------------------------------------

\section{Background}
\label{sec: background}

Structures can have numerous elements and interacting parts.
Each element adds to the system's degrees of freedom (DOFs).
As structures become more complex, their dynamic analysis becomes increasingly expensive because one must simultaneously compute more DOFs.
Component mode synthesis (CMS) addresses this challenge by dividing the structure into smaller components and models the dynamic behavior and interactions of these components \cite{hurty1960vibrations}\cite{hurty1965dynamic}.
This component-wise analysis is faster and more cost-effective than evaluating the complete structure simultaneously.

As the benefit of CMS became apparent, further development followed, which led to the classic Craig-Bampton (CB) method in \cite{craig1968coupling}.
The method distinguishes the components' internal dynamic responses from intercomponent interactions via a modal decomposition.
By performing mathematical reduction on the internal dynamic model, the CB method reduces the number of DOFs and, therefore, the computational cost.
However, the reduction introduces a discrepancy between the assembly of these components and the original structure.
Consequently, there have been several efforts to improve the accuracy of the CB method, such as the flexibility-based CMS (F-CMS) \cite{park2004partitioned} and the enhanced CB (ECB) method \cite{kim2015enhanced}\cite{boo2018towards}.
Regardless, the CB method is still popular because of its simplicity.

Like most available approaches, the CB method assumes that all structural parameters are precisely known.
In reality, this is seldom the case.
Therefore, uncertainty quantification due to variability in the structural parameters is necessary.
Several publications have combined CMS and the CB method with uncertainty quantification methods, such as the perturbation method \cite{hinke2009component}\cite{sarsri2016dynamic}, the stochastic reduced basis projection \cite{dohnal2009joint}, and neural networks \cite{chatterjee2021multilevel}.

In a separate context, surrogate modeling has emerged as a popular method for uncertainty quantification in structural dynamics.
An example of a surrogate model is the polynomial chaos expansion (PCE), which approximates model outputs using linear combinations of multivariate polynomials.
There are various means to obtain the polynomials' coefficients, including the stochastic Galerkin method \cite{Ghanem1991}, numerical quadrature \cite{xiu2007efficient}, interpolation \cite{babuvska2007stochastic}, and various regression techniques \cite{berveiller2006stochastic}\cite{blatman2010adaptive}\cite{blatman2011adaptive}.

Despite its popularity, PCE performs poorly in approximating frequency response functions around structures' eigenfrequencies \cite{jacquelin2015polynomial}.
Authors of \cite{jacquelin2017polynomial} proposed to approximate model outputs using ratios of PCEs (RPCE) instead, using the stochastic Galerkin method to compute the PCEs' coefficients.
The authors of the paper show that RPCE requires a lower number of basis functions to approximate highly nonlinear models compared to PCE.

Recently, the authors of \cite{schneider2020polynomial} developed the non-intrusive RPCE (NI-RPCE), using regression to compute the coefficients.
This model saw further development in \cite{schneider2023sparse}, in which sparse Bayesian learning helps to avoid overfitting.
An attractive feature of NI-RPCE is that it approximates each output independently.
Therefore, the CB method and NI-RPCE combination may be promising for uncertainty propagation as the surrogate model can approximate each component's dynamic response separately.