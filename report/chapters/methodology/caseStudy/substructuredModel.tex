% -----------------------------------------------------------------------------
% Master thesis in the study program computational mechanics
%
% B.Sc. Rezha Adrian Tanuharja - 03751261
% M.Sc. Felix Schneider (supervisor)
%
% chapters/methodology/caseStudy/substructuredModel.tex
% Last edited 03 November 2023
% -----------------------------------------------------------------------------

\subsection{Substructured Plate Model}
\label{ssec: substructured model}

The substructured plate model partitions the plate into three components along the $x$ axis.
The components have different numbers of DOFs and different modes because their lengths are not equal.
This choice is intentional, since in practical applications, components rarely have the same number of DOFs and the same modes.
Table \ref{table: components' parameters} provides the meshes' parameters for the three components.

\begin{table}[H]
    \setlength{\extrarowheight}{2pt}
    \centering
    \begin{tabular}{
        L{0.3\textwidth} C{0.15\textwidth} C{0.15\textwidth} C{0.15\textwidth}
    }
    \hline
    Parameter & Left & Middle & Right \\
    \hline
    Number of element (x) &
    16 & 14 & 18 \\
    Number of element (y) &
    20 & 20 & 20 \\
    & & & \\
    Number of Nodes &
    1353 & 1189 & 1517 \\
    Number of DOFs &
    4059 & 3567 & 4551 \\
    \hline
    \end{tabular}
    \caption{Substructured Plate Model's Meshes' Parameters}
    \label{table: components' parameters}
\end{table}

The author selects the meshes' parameters such that the elements' sizes are uniform and equal to the elements' sizes in the complete plate model.
This is to avoid numerical artifacts in the comparison caused by different elements' sizes between the two models.
Figure \ref{fig: substructured model} illustrates the substructured plate model and the two nodes.

% -----------------------------------------------------------------------------
% Master thesis in the study program computational mechanics
%
% B.Sc. Rezha Adrian Tanuharja - 03751261
% M.Sc. Felix Schneider (supervisor)
%
% chapters/methodology/caseStudy/substructuredModel.tex
% Last edited 03 November 2023
% -----------------------------------------------------------------------------

\subsection{Substructured Plate Model}
\label{ssec: substructured model}

The substructured plate model partitions the plate into three components along the $x$ axis.
The components have different numbers of DOFs and different modes because their lengths are not equal.
This choice is intentional, since in practical applications, components rarely have the same number of DOFs and the same modes.
Table \ref{table: components' parameters} provides the meshes' parameters for the three components.

\begin{table}[H]
    \setlength{\extrarowheight}{2pt}
    \centering
    \begin{tabular}{
        L{0.3\textwidth} C{0.15\textwidth} C{0.15\textwidth} C{0.15\textwidth}
    }
    \hline
    Parameter & Left & Middle & Right \\
    \hline
    Number of element (x) &
    16 & 14 & 18 \\
    Number of element (y) &
    20 & 20 & 20 \\
    & & & \\
    Number of Nodes &
    1353 & 1189 & 1517 \\
    Number of DOFs &
    4059 & 3567 & 4551 \\
    \hline
    \end{tabular}
    \caption{Substructured Plate Model's Meshes' Parameters}
    \label{table: components' parameters}
\end{table}

The author selects the meshes' parameters such that the elements' sizes are uniform and equal to the elements' sizes in the complete plate model.
This is to avoid numerical artifacts in the comparison caused by different elements' sizes between the two models.
Figure \ref{fig: substructured model} illustrates the substructured plate model and the two nodes.

% -----------------------------------------------------------------------------
% Master thesis in the study program computational mechanics
%
% B.Sc. Rezha Adrian Tanuharja - 03751261
% M.Sc. Felix Schneider (supervisor)
%
% chapters/methodology/caseStudy/substructuredModel.tex
% Last edited 03 November 2023
% -----------------------------------------------------------------------------

\subsection{Substructured Plate Model}
\label{ssec: substructured model}

The substructured plate model partitions the plate into three components along the $x$ axis.
The components have different numbers of DOFs and different modes because their lengths are not equal.
This choice is intentional, since in practical applications, components rarely have the same number of DOFs and the same modes.
Table \ref{table: components' parameters} provides the meshes' parameters for the three components.

\begin{table}[H]
    \setlength{\extrarowheight}{2pt}
    \centering
    \begin{tabular}{
        L{0.3\textwidth} C{0.15\textwidth} C{0.15\textwidth} C{0.15\textwidth}
    }
    \hline
    Parameter & Left & Middle & Right \\
    \hline
    Number of element (x) &
    16 & 14 & 18 \\
    Number of element (y) &
    20 & 20 & 20 \\
    & & & \\
    Number of Nodes &
    1353 & 1189 & 1517 \\
    Number of DOFs &
    4059 & 3567 & 4551 \\
    \hline
    \end{tabular}
    \caption{Substructured Plate Model's Meshes' Parameters}
    \label{table: components' parameters}
\end{table}

The author selects the meshes' parameters such that the elements' sizes are uniform and equal to the elements' sizes in the complete plate model.
This is to avoid numerical artifacts in the comparison caused by different elements' sizes between the two models.
Figure \ref{fig: substructured model} illustrates the substructured plate model and the two nodes.

% -----------------------------------------------------------------------------
% Master thesis in the study program computational mechanics
%
% B.Sc. Rezha Adrian Tanuharja - 03751261
% M.Sc. Felix Schneider (supervisor)
%
% chapters/methodology/caseStudy/substructuredModel.tex
% Last edited 03 November 2023
% -----------------------------------------------------------------------------

\subsection{Substructured Plate Model}
\label{ssec: substructured model}

The substructured plate model partitions the plate into three components along the $x$ axis.
The components have different numbers of DOFs and different modes because their lengths are not equal.
This choice is intentional, since in practical applications, components rarely have the same number of DOFs and the same modes.
Table \ref{table: components' parameters} provides the meshes' parameters for the three components.

\begin{table}[H]
    \setlength{\extrarowheight}{2pt}
    \centering
    \begin{tabular}{
        L{0.3\textwidth} C{0.15\textwidth} C{0.15\textwidth} C{0.15\textwidth}
    }
    \hline
    Parameter & Left & Middle & Right \\
    \hline
    Number of element (x) &
    16 & 14 & 18 \\
    Number of element (y) &
    20 & 20 & 20 \\
    & & & \\
    Number of Nodes &
    1353 & 1189 & 1517 \\
    Number of DOFs &
    4059 & 3567 & 4551 \\
    \hline
    \end{tabular}
    \caption{Substructured Plate Model's Meshes' Parameters}
    \label{table: components' parameters}
\end{table}

The author selects the meshes' parameters such that the elements' sizes are uniform and equal to the elements' sizes in the complete plate model.
This is to avoid numerical artifacts in the comparison caused by different elements' sizes between the two models.
Figure \ref{fig: substructured model} illustrates the substructured plate model and the two nodes.

\input{images/substructuredModel}