% -----------------------------------------------------------------------------
% Master thesis in the study program computational mechanics
%
% B.Sc. Rezha Adrian Tanuharja - 03751261
% M.Sc. Felix Schneider (supervisor)
%
% images/procedureUCB.tex
% Last edited 03 November 2023
% -----------------------------------------------------------------------------

\begin{figure}[H]
    \centering
    \scalebox{0.75}{
        \begin{tikzpicture}[
            startstop/.style={%
                thick, rectangle, rounded corners,%
                minimum width=\charthgap, minimum height=\chartvgap,%
                text centered, draw=black, fill=white, text width=\charttwid%
            },
            io/.style={%
                thick, trapezium, trapezium left angle=75, trapezium right angle=105,%
                minimum width=\charthgap, minimum height=\chartvgap,%
                text centered, draw=black, fill=white, text width=\chartvgap%
            },
            process/.style={%
                thick, rectangle,%
                minimum width=\charthgap, minimum height=\chartvgap,%
                text centered, draw=black, fill=white, text width=\charttwid%
            },
            blankprocess/.style={%
                thick, rectangle,%
                minimum width=\charthgap, minimum height=\chartvgap,%
                text centered, draw=none, fill=none, text width=\charttwid%
            },
            decision/.style={%
                thick, diamond,%
                minimum width=\chartvgap, minimum height=\chartvgap,%
                text centered, draw=black, fill=white, text width=\chartvgap%
            },
            arrow/.style={thick,->,>=stealth}
        ]

        % Define nodes
        \node (start) [startstop] {%
            Start
        };

        \node (input1) [
                io, below of=start,
                % xshift=-0.7*\charthgap,
                yshift=-\chartvgap
            ] {$\phantom{a}$};

        \node (ED_1) [
                blankprocess, below of=start,
                % xshift=-0.7*\charthgap,
                yshift=-\chartvgap
            ] {
            Experimental Design
            };

        \node (fullEval) [
                process, below of=ED_1,
                xshift=-0.7*\charthgap,
                yshift=-\chartvgap
            ] {
                MC Simulations\\
                Complete Plate Model
            };

        \node (subEval) [
                process, below of=ED_1,
                xshift=0.7*\charthgap,
                yshift=-\chartvgap
            ] {
                MC Simulations\\
                CUCB / HUCB
            };

        \node (input5) [io, below of=subEval, yshift=-\chartvgap] {
            $\phantom{a}$
        };
        \node (ED_5) [blankprocess, below of=subEval, yshift=-\chartvgap] {
            Output Estimates\\
            CUCB / HUCB
        };

        \node (input6) [io, below of=fullEval, yshift=-\chartvgap] {
            $\phantom{a}$
        };
        \node (ED_6) [blankprocess, below of=fullEval, yshift=-\chartvgap] {
            Output Benchmarks
        };

        \path 
            (ED_5) -- (ED_6) coordinate[midway] (ED_56);

        \node (compare) [process, below of=ED_56, yshift=-\chartvgap] {
            Bootstrapping
        };
        \node (iterate) [process, below of=compare, yshift=-\chartvgap] {
            Comparison
        };
        \node (stop) [startstop, below of=iterate, yshift=-1.2*\chartvgap] {Stop};

        \draw [arrow] (start) -- (ED_1);
        \draw [arrow] (input1) -| (fullEval);
        \draw [arrow] (input1) -| (subEval);
        \draw [arrow] (fullEval) -- (ED_6);
        \draw [arrow] (subEval) -- (ED_5);
        \draw [arrow] (ED_6) |- (compare);
        \draw [arrow] (ED_5) |- (compare);
        \draw [arrow] (compare) -- (iterate);
        \draw [arrow] (iterate) -- (stop);

        % \path (start) -- (ED_1) coordinate[midway] (midpoint_1);
        % \draw [arrow] (midpoint_1) -| (ED_2);
        % \draw [arrow] (ED_2) -- (subEval);
        % \draw [arrow] (subEval) -- (ED_3);
        % \draw [arrow] (ED_3) -- (resample);
        % \draw [arrow] (resample) -- (ED_4);
        % \draw [arrow] (ED_4) -- (train);
        % \draw [arrow] (train) -- (evalRPCE);
        % \draw [arrow] (evalRPCE) -- (ED_5);
        % \draw [arrow] (ED_5) |- (compare);

        % \draw [arrow] (compare) -- (iterate);
        % \draw [arrow] (iterate) -- node[right] {n} (stop) ;

        % \draw [thick] 
        %     (iterate) -- 
        %     ++(1.2*\chartvgap,0) node[midway, above] {y} coordinate (endpoint_1);
        % \draw [thick] 
        %     (endpoint_1) -- node[above] {$i=i+1$}
        %     ++(1.0*\charthgap,0) coordinate (endpoint_2);

        % \draw [arrow] 
        %     (endpoint_2) |- (resample);

        % \path 
        %     (ED_1) -- (fullEval) coordinate[midway] (point_1);
        % \path 
        %     (ED_2) -- (subEval) coordinate[midway] (point_2);
        % \path 
        %     (point_1) -- (point_2) coordinate[midway] (point_3);
        % \draw [thick] 
        %     (point_1) -- (point_3);
        % \draw [arrow] 
        %     (point_3) |- (evalRPCE);

        \end{tikzpicture}
    }
    \caption{The UCBs Evaluation Procedure Flowchart}
    \label{flowchart: UCB procedure}
\end{figure}