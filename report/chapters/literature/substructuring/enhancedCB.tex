% -----------------------------------------------------------------------------
% Master thesis in the study program computational mechanics
%
% B.Sc. Rezha Adrian Tanuharja - 03751261
% M.Sc. Felix Schneider (supervisor)
%
% chapters/literature/substructuring/enhancedCB.tex
% Last edited 03 November 2023
% -----------------------------------------------------------------------------

\subsection{Enhanced Craig-Bampton Method}
\label{ssec: ECB method}

The CB method uses only the dominant modes and neglects the contribution of the rest of the normal modes, referred to as the residual modes.
One approach to improve its accuracy is to compensate for the absence of these residual modes in the formulation.
{The ECB method does this by considering the residual modes in the transformation $\mathbf{U}=\tilde{\mathbf{T}}\mathbf{X}$}%
\begin{equation}
    \begin{pmatrix}
        \mathbf{U}_{i} \\
        \mathbf{U}_{b}
    \end{pmatrix}
    =
    \begin{bmatrix}
        \;\;\mathbf{\Phi}_{d}
        & 
        \;\;\mathbf{\Phi}_{r}
        &
        \;\;\mathbf{\Phi}_{c} \\
        \mathbf{0} & \mathbf{0} & \mathbf{I}
    \end{bmatrix}
    \begin{pmatrix}
        \mathbf{X}_{d} \\
        \mathbf{X}_{r} \\
        \mathbf{U}_{b}
    \end{pmatrix}.
    \label{enhanced_transformation}
\end{equation}
{Using $\tilde{\mathbf{T}}$ from \eqref{enhanced_transformation} in the transformation $\overline{\mathbf{D}}=\tilde{\mathbf{T}}^{T}\mathbf{D}\tilde{\mathbf{T}}$, the second row of the dynamic stiffness becomes}%
\begin{equation}
    \begin{bmatrix}
        \overline{\mathbf{D}}_{rd} &
        \overline{\mathbf{D}}_{rr} &
        \overline{\mathbf{D}}_{rc} 
    \end{bmatrix}
    =
    \begin{bmatrix}
        \mathbf{0}
        &
        \mathbf{\Phi}_{r}^{T}
        \mathbf{D}_{ii}
        \mathbf{\Phi}_{r}
        \quad
        &
        \mathbf{\Phi}_{r}^{T}
        \mathbf{D}_{ii}
        \mathbf{\Phi}_{b}
        +
        \mathbf{\Phi}_{r}^{T}
        \mathbf{D}_{ib}
    \end{bmatrix}.
\end{equation}
{Imposing the condition of zero residual dynamic force, the relation between the residual modal DOFs and the boundary DOFs is}%
\begin{equation}
    \mathbf{X}_{r}
    =
    -\mathbf{\Lambda}_{r}^{-1}
    \left[
        \mathbf{\Phi}_{r}^{T}
        \mathbf{D}_{ii}
        \mathbf{\Phi}_{b}
    +
        \mathbf{\Phi}_{r}^{T}
        \mathbf{D}_{ib}
    \right]
    \mathbf{U}_{b},
    \text{ where }
    \mathbf{\Lambda}_{r}
    =
    \mathbf{\Phi}_{r}^{T}
    \mathbf{D}_{ii}
    \mathbf{\Phi}_{r}.
    \label{residual_and_boundary_relation}
\end{equation}
{Subsequently, ECB uses \eqref{residual_and_boundary_relation} to condense \eqref{enhanced_transformation} into}%
\begin{equation}
    \begin{pmatrix}
        \mathbf{U}_{i} \\
        \mathbf{U}_{b}
    \end{pmatrix}
    =
    \begin{bmatrix}
        \mathbf{\Phi}_{d} &
        -\mathbf{D}_{ii}^{-1} \mathbf{D}_{ib}
        +
        \left(
            \mathbf{\Phi}_{d}
            \mathbf{\Lambda}_{d}^{-1}
            \mathbf{\Phi}_{d}^{T}
        \right)
        \left(
            \mathbf{D}_{ii}
            \mathbf{\Phi}_{c}
            +
            \mathbf{D}_{ib}
        \right)
        \\
        \mathbf{0}
        &
        \mathbf{I}
    \end{bmatrix}
    \begin{pmatrix}
        \mathbf{X}_{d} \\
        \mathbf{U}_{b}
    \end{pmatrix},
    \label{final_enhanced_transformation}
\end{equation}
in which $
    \mathbf{\Lambda}_{d}
    =
    \mathbf{\Phi}_{d}^{T}
    \mathbf{D}_{ii}
    \mathbf{\Phi}_{d}
$.
Using \eqref{final_enhanced_transformation} instead of \eqref{cb_transformation} improves the CB method's accuracy.
Further simplifications and detailed steps to compute the transformation efficiently are available in \cite{kim2015enhanced}, but are not relevant to this study.