% -----------------------------------------------------------------------------
% Master thesis in the study program computational mechanics
%
% B.Sc. Rezha Adrian Tanuharja - 03751261
% M.Sc. Felix Schneider (supervisor)
%
% chapters/literature/substructuring/solutionCB.tex
% Last edited 03 November 2023
% -----------------------------------------------------------------------------

\subsection{Assembly and Solution Procedure}
\label{ssec: solution}

The following assembly approach uses the primal formulation in \cite{de2008general}.
{Let $\mathbf{U}_{B}$ be a vector containing all unique boundary DOFs and $\mathbf{L}^{(r)}$ be a binary matrix localizing the boundary DOFs of the $r$-th component such that $\mathbf{U}_{b}^{(r)}=\mathbf{L}^{(r)}\mathbf{U}_{B}$ and}%
\begin{equation}
    \begin{pmatrix}
        \mathbf{X}_{d}^{(r)} \\
        \mathbf{U}_{b}^{(r)}
    \end{pmatrix}
    =
    \begin{bmatrix}
        \mathbf{I} & \mathbf{0} \\
        \mathbf{0} & \mathbf{L}^{(r)}
    \end{bmatrix}
    \begin{pmatrix}
        \mathbf{X}_{d}^{(r)} \\
        \mathbf{U}_{B}
    \end{pmatrix}.
    \label{assembly_transformation}
\end{equation}
{Using the transformation \eqref{assembly_transformation}, the reduced component's dynamic equation of motion in the frequency domain \eqref{modal_eq_of_motion} becomes}%
\begin{equation}
    \begin{bmatrix}
        \overline{\mathbf{D}}_{dd}^{(r)} &
        \overline{\mathbf{D}}_{dB}^{(r)} \\
        \overline{\mathbf{D}}_{Bd}^{(r)} &
        \overline{\mathbf{D}}_{BB}^{(r)}
    \end{bmatrix}
    \begin{pmatrix}
        \mathbf{X}_{d}^{(r)} \\
        \mathbf{U}_{B}
    \end{pmatrix}
    =
    \begin{pmatrix}
        \overline{\mathbf{F}}_{d}^{(r)} \\
        \overline{\mathbf{F}}_{B}^{(r)}
    \end{pmatrix}
    +
    \begin{pmatrix}
        \mathbf{0} \\
        \mathbf{G}_{B}^{(r)}
    \end{pmatrix},
    \label{semi_global_dynamic_equation}
\end{equation}
{where}%
\begin{equation}
    \begin{bmatrix}
        \overline{\mathbf{D}}_{dd}^{(r)} &
        \overline{\mathbf{D}}_{dB}^{(r)} \\
        \overline{\mathbf{D}}_{Bd}^{(r)} &
        \overline{\mathbf{D}}_{BB}^{(r)}
    \end{bmatrix}
    =
    \begin{bmatrix}
        \mathbf{I} & \mathbf{0} \\
        \mathbf{0} & \mathbf{L}^{(r)}
    \end{bmatrix}^{T}
    \begin{bmatrix}
        \overline{\mathbf{D}}_{dd}^{(r)} &
        \overline{\mathbf{D}}_{dc}^{(r)} \\
        \overline{\mathbf{D}}_{cd}^{(r)} &
        \overline{\mathbf{D}}_{bc}^{(r)}
    \end{bmatrix}
    \begin{bmatrix}
        \mathbf{I} & \mathbf{0} \\
        \mathbf{0} & \mathbf{L}^{(r)}
    \end{bmatrix},
\end{equation}
\vspace{-1.0em}
\begin{equation}
    \begin{pmatrix}
        \overline{\mathbf{F}}_{d}^{(r)} \\
        \overline{\mathbf{F}}_{B}^{(r)}
    \end{pmatrix}
    =
    \begin{bmatrix}
        \mathbf{I} & \mathbf{0} \\
        \mathbf{0} & \mathbf{L}^{(r)}
    \end{bmatrix}^{T}
    \begin{pmatrix}
        \overline{\mathbf{F}}_{d}^{(r)} \\
        \overline{\mathbf{F}}_{c}^{(r)}
    \end{pmatrix},
\end{equation}
{and}%
\begin{equation}
    \begin{pmatrix}
        \mathbf{0} \\
        \mathbf{G}_{B}^{(r)}
    \end{pmatrix}
    =
    \begin{bmatrix}
        \mathbf{I} & \mathbf{0} \\
        \mathbf{0} & \mathbf{L}^{(r)}
    \end{bmatrix}^{T}
    \begin{pmatrix}
        \mathbf{0} \\
        \mathbf{G}_{b}^{(r)}
    \end{pmatrix}.
\end{equation}
The strong equilibrium condition requires the sum of interaction forces to be zero, i.e., $\sum_{r}\mathbf{G}_{B}^{(r)}=\mathbf{0}$.
{Consequently, the sum of all boundary forces in \eqref{semi_global_dynamic_equation} is}%
\begin{equation}
    \sum_{r}{
        \overline{\mathbf{F}}_{B}^{(r)}
    }
    =
    \sum_{r}{\left(
        \overline{\mathbf{D}}_{Bd}^{(r)}
        \mathbf{X}_{i}^{(r)}
        +
        \overline{\mathbf{D}}_{BB}^{(r)}
        \mathbf{U}_{B}
    \right)}.
    \label{boundary_force_sum}
\end{equation}
Equation \eqref{semi_global_dynamic_equation} also gives the relation between the component's internal DOFs and the boundary DOFs:
\begin{equation}
    \overline{\mathbf{F}}_{d}^{(r)}
    =
    \overline{\mathbf{D}}_{dd}^{(r)}
    \mathbf{X}_{d}^{(r)}
    +
    \overline{\mathbf{D}}_{dB}^{(r)}
    \mathbf{U}_{B}
    \iff
    \mathbf{X}_{d}^{(r)}
    =
    \left(
        \overline{\mathbf{D}}_{dd}^{(r)}
    \right)^{-1}
    \left(
        \overline{\mathbf{F}}_{d}^{(r)}
        -
        \overline{\mathbf{D}}_{dB}^{(r)}
        \mathbf{U}_{B}
    \right).
    \label{internal_boundary_relation}
\end{equation}
The substitution of \eqref{internal_boundary_relation} into \eqref{boundary_force_sum} yields
\begin{equation}
    \sum_{r}{\left(
        \overline{\mathbf{F}}_{B}^{(r)}
        -
        \overline{\mathbf{D}}_{Bd}^{(r)}
        \left(
            \overline{\mathbf{D}}_{dd}^{(r)}
        \right)^{-1}
        \overline{\mathbf{F}}_{d}^{(r)}
    \right)}
    =
    \sum_{r}{\left(
        \overline{\mathbf{D}}_{BB}^{(r)}
        -
        \overline{\mathbf{D}}_{Bd}^{(r)}
        \left(
            \overline{\mathbf{D}}_{dd}^{(r)}
        \right)^{-1}
        \overline{\mathbf{D}}_{dB}^{(r)}
    \right)}
    \mathbf{U}_{B}.
    \label{schur_complement}
\end{equation}
Solving \eqref{internal_boundary_relation} and \eqref{schur_complement} gives the dynamic response of the assembled components.
Algorithm \ref{alg: solution procedure} illustrates a solution procedure with parallel processing.
% -----------------------------------------------------------------------------
% Master thesis in the study program computational mechanics
%
% B.Sc. Rezha Adrian Tanuharja - 03751261
% M.Sc. Felix Schneider (supervisor)
%
% algorithm/solutionProcedure.tex
% Last edited 03 November 2023
% -----------------------------------------------------------------------------

\begin{center}
\begin{algorithm}[H]
    \label{alg: solution procedure}
    \SetKwInOut{Input}{Input}
    \SetKwInOut{Output}{Output\,}
    \mpiInit{} \\
    $\dots$ \\
    \vspace{1.0em}%
    \tcp{All components compute summands of \eqref{schur_complement} in parallel}
    $
        \mathbf{S}_{B\text{local}} \longleftarrow
        \overline{\mathbf{D}}_{BB}
        -
        \overline{\mathbf{D}}_{Bd}
        \left(
            \overline{\mathbf{D}}_{dd}
        \right)^{-1}
        \overline{\mathbf{D}}_{dB}
    $ \\
    $
        \mathbf{F}_{B\text{local}} \longleftarrow
        \overline{\mathbf{F}}_{B}
        -
        \overline{\mathbf{D}}_{Bd}
        \left(
            \overline{\mathbf{D}}_{dd}
        \right)^{-1}
        \overline{\mathbf{F}}_{d}
    $ \\
    \vspace{1.0em}%
    \tcp{Gather matrices and vectors in the central process}
    \uIf{rank $== 0$ }{
        $\mathbf{S}_{B} \longleftarrow$
        \mpiGather{$\mathbf{S}_{B\text{local}}$} \\
        $\mathbf{F}_{B} \longleftarrow$
        \mpiGather{$\mathbf{F}_{B\text{local}}$} \\
        \vspace{1.0em}%
        \tcp{Solve for boundary DOFs and send to all components}
        $\mathbf{U}_{B} \longleftarrow$
        $($\sum{$\mathbf{S}_{B}$}$)^{-1}$
        \sum{$\mathbf{F}_{B}$} \\
        \vspace{1.0em}%
        \mpiBroadcast{$\mathbf{U}_{B}$}
    }
    \Else{
        \tcc{Do nothing}
    }
    \vspace{1.0em}%
    \tcp{All components solve for internal DOFs in parallel}
    $
        \mathbf{X}_{d} \longleftarrow
        \overline{\mathbf{D}}_{dd}^{-1}
        \left(
            \overline{\mathbf{F}}_{d}
            -
            \overline{\mathbf{D}}_{dB}
            \mathbf{U}_{B}
        \right)
    $ \\
    \vspace{1.0em}%
    $\dots$ \\
    \mpiFinalize{}
  \caption{Parallel Solution of Coupled Dynamic Components}
\end{algorithm}
\end{center}