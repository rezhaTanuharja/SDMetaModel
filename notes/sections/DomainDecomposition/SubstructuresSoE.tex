\documentclass[../DomainDecomposition.tex]{subfiles} 
\bibliography{../../literature}

\begin{document} 

\subsection{Substructures' System of Equations}

This part is adapted from \cite{craig1968coupling}. 
\vspace{10pt} 

The degree of freedoms associated with shared nodes (between two or more substructures) are referred to as boundary DoFs while other are referred to as internal DoFs. 
The static system of equations for substructure $r$ is given by the following: 

\begin{equation}
    \begin{bmatrix}
        \mathbf{F}_{r}^{I} \\ 
        \mathbf{F}_{r}^{B} 
    \end{bmatrix}
    =
    \begin{bmatrix}
        \mathbf{K}_{r}^{II} && 
        \mathbf{K}_{r}^{IB} \\ 
        \mathbf{K}_{r}^{BI} && 
        \mathbf{K}_{r}^{BB} 
    \end{bmatrix} 
    \begin{bmatrix}
        \mathbf{u}_{r}^{I} \\ 
        \mathbf{u}_{r}^{B} 
    \end{bmatrix} 
    \label{Partitioned SoE} 
\end{equation}

Constraint modes are defined as the mode shapes of internal freedoms due to successive unit displacement of boundary points, all other boundary point being totally constrained. 
Setting forces associated with internal DoFs to zero in equation \eqref{Partitioned SoE}: 

\begin{equation*}
    \mathbf{0} 
    =
    \mathbf{K}_{r}^{IB} \mathbf{u}_{r}^{B} + 
    \mathbf{K}_{r}^{II} \mathbf{u}_{r}^{I} + 
    \phantom{x}
    \implies
    \phantom{x}
    \mathbf{u}_{r}^{I} 
    =
    -\left(\mathbf{K}_{r}^{II}\right)^{-1} 
    \mathbf{K}_{r}^{IB} \mathbf{u}_{r}^{B} 
\end{equation*}

From which an expression for constrained modes is obtained: 

\begin{equation}
    \bar{\mathbf{\Phi}}_{r}^{C}
    =
    -\left(\mathbf{K}_{r}^{II}\right)^{-1} 
    \mathbf{K}_{r}^{IB} 
\end{equation}

Normal modes are defined as the mode shapes of the substructure with totally constrained boundary. 
These are obtained from the equations 

\begin{equation}
    \mathbf{0} 
    = 
    \left[
        \mathbf{K}_{r}^{II} - 
        \omega_{j}^{2} \mathbf{M}_{r}^{II} 
    \right]
    \mathbf{\Phi}_{r,j}^{I}
    \label{Internal Eigen Equations}
\end{equation}

The eigenvectors of equation \eqref{Internal Eigen Equations} form the columns of the normal mode matrix $\mathbf{\Phi}_{r}^{N}$. 
Model order reduction is possible through truncation of this matrix's columns i.e. by retaining only some eigenvectors in the reduced normal mode matrix $\bar{\mathbf{\Phi}}_{r}^{N}$. 
Transformation between freedoms in physical space and modal space is given by the following: 

\begin{equation}
    \mathbf{u}_{r} 
    = 
    \mathbf{G}_{r} 
    \bar{\mathbf{u}}_{r}
    \phantom{xx} 
    \text{or} 
    \phantom{xx} 
    \begin{bmatrix}
        \mathbf{u}_{r}^{I} \\
        \mathbf{u}_{r}^{B} 
    \end{bmatrix}
    =
    \begin{bmatrix}
        \bar{\mathbf{\Phi}}_{r}^{N} &
        \bar{\mathbf{\Phi}}_{r}^{C} \\ 
        \mathbf{0} & 
        \mathbf{I} 
    \end{bmatrix}
    \begin{bmatrix}
        \bar{\mathbf{u}}_{r}^{N} \\ 
        \bar{\mathbf{u}}_{r}^{C}
    \end{bmatrix}
\end{equation}

The modal mass and stiffness matrices are given by the followings respectively 

\begin{equation}
    \bar{\mathbf{M}}_{r} 
    =
    \mathbf{G}_{r}^{T} \mathbf{M}_{r} \mathbf{G}_{r}
    \phantom{xx} 
    \text{or} 
    \phantom{xx} 
    \begin{bmatrix}
        \bar{\mathbf{M}}_{r}^{NN} & 
        \bar{\mathbf{M}}_{r}^{NB} \\
        \bar{\mathbf{M}}_{r}^{BN} & 
        \bar{\mathbf{M}}_{r}^{BB} 
    \end{bmatrix}
    =
    \begin{bmatrix}
        \bar{\mathbf{\Phi}}_{r}^{N} &
        \bar{\mathbf{\Phi}}_{r}^{C} \\ 
        \mathbf{0} & 
        \mathbf{I} 
    \end{bmatrix}^{T} 
    \begin{bmatrix}
        \mathbf{M}_{r}^{II} & 
        \mathbf{M}_{r}^{IB} \\
        \mathbf{M}_{r}^{BI} & 
        \mathbf{M}_{r}^{BB} 
    \end{bmatrix}
    \begin{bmatrix}
        \bar{\mathbf{\Phi}}_{r}^{N} &
        \bar{\mathbf{\Phi}}_{r}^{C} \\ 
        \mathbf{0} & 
        \mathbf{I} 
    \end{bmatrix}
\end{equation}

\begin{equation}
    \bar{\mathbf{K}}_{r} 
    =
    \mathbf{G}_{r}^{T} \mathbf{K}_{r} \mathbf{G}_{r}
    \phantom{xx} 
    \text{or} 
    \phantom{xx} 
    \begin{bmatrix}
        \bar{\mathbf{K}}_{r}^{NN} & 
        \bar{\mathbf{K}}_{r}^{NB} \\
        \bar{\mathbf{K}}_{r}^{BN} & 
        \bar{\mathbf{K}}_{r}^{BB} 
    \end{bmatrix}
    =
    \begin{bmatrix}
        \bar{\mathbf{\Phi}}_{r}^{N} &
        \bar{\mathbf{\Phi}}_{r}^{C} \\ 
        \mathbf{0} & 
        \mathbf{I} 
    \end{bmatrix}^{T} 
    \begin{bmatrix}
        \mathbf{K}_{r}^{II} & 
        \mathbf{K}_{r}^{IB} \\
        \mathbf{K}_{r}^{BI} & 
        \mathbf{K}_{r}^{BB} 
    \end{bmatrix}
    \begin{bmatrix}
        \bar{\mathbf{\Phi}}_{r}^{N} &
        \bar{\mathbf{\Phi}}_{r}^{C} \\ 
        \mathbf{0} & 
        \mathbf{I} 
    \end{bmatrix}
\end{equation}

The modal stiffness matrix is block-diagonal because 

\begin{equation*}
    \begin{aligned}
        \bar{\mathbf{K}}_{r}^{NB} 
        &= 
        \left[\bar{\mathbf{\Phi}}_{r}^{N} \right]^{T} \mathbf{K}_{r}^{IB} + 
        \left[\bar{\mathbf{\Phi}}_{r}^{N} \right]^{T} \mathbf{K}_{r}^{II} 
              \bar{\mathbf{\Phi}}_{r}^{C} \\
        &=
        \left[\bar{\mathbf{\Phi}}_{r}^{N} \right]^{T} \mathbf{K}_{r}^{IB} + 
        \left[\bar{\mathbf{\Phi}}_{r}^{N} \right]^{T} \mathbf{K}_{r}^{II} 
            \left[
                -\left(\mathbf{K}_{r}^{II}\right)^{-1} 
                \mathbf{K}_{r}^{IB} 
            \right] \\
        &= 
        \mathbf{0} 
    \end{aligned}
\end{equation*}

The modal load vector is given by the following equation

\begin{equation}
    \bar{\mathbf{F}}_{r} 
    =
    \mathbf{G}_{r}^{T} \mathbf{F}_{r} 
    \phantom{xx} 
    \text{or} 
    \phantom{xx} 
    \begin{bmatrix}
        \bar{\mathbf{F}}_{r}^{N} \\
        \bar{\mathbf{F}}_{r}^{C} \\
    \end{bmatrix}
    =
    \begin{bmatrix}
        \bar{\mathbf{\Phi}}_{r}^{N} &
        \bar{\mathbf{\Phi}}_{r}^{C} \\ 
        \mathbf{0} & 
        \mathbf{I} 
    \end{bmatrix}^{T} 
    \begin{bmatrix}
        \mathbf{F}_{r}^{I} \\
        \mathbf{F}_{r}^{B} \\
    \end{bmatrix}
\end{equation}

\end{document} 