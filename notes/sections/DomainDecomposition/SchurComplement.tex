\documentclass[../DomainDecomposition.tex]{subfiles} 
\bibliography{../../literature}

\begin{document}

\subsection{Schur Complement} 

This part is adapted from \cite{chatterjee2021multilevel}. 
\vspace{10pt} 

The idea of Schur complement is to solve the system of equations in two steps. 
In the first step, condensation is performed and the resulting equation is solved for the constrained modes. 
Subsequently, the constrained modes are substituted to the remaining equations to solve for the internal modes. 
\vspace{12pt} 

The formula for internal modes' forces are obtained from equation \eqref{Modal SoE} and are given by the following: 

\begin{equation*}
    \bar{\mathbf{F}}_{\alpha}^{N} 
    =
    \bar{\mathbf{D}}_{\alpha}^{NN} \bar{\mathbf{u}}_{\alpha}^{N} 
    +
    \bar{\mathbf{D}}_{\alpha}^{NC} \bar{\mathbf{u}}^{C} 
    \phantom{xx} 
    \text{and} 
    \phantom{xx} 
    \bar{\mathbf{F}}_{\beta}^{N} 
    =
    \bar{\mathbf{D}}_{\beta}^{NN} \bar{\mathbf{u}}_{\beta}^{N} 
    +
    \bar{\mathbf{D}}_{\beta}^{NC} \bar{\mathbf{u}}^{C} 
\end{equation*}

To perform condensation, expressions of the internal modes are required. 
Therefore, the two equations are rearranged and solved for the internal modes respectively: 

\begin{equation}
    \bar{\mathbf{u}}_{\alpha}^{N} 
    =
    \left(
        \bar{\mathbf{D}}_{\alpha}^{NN}
    \right)^{-1} 
    \bar{\mathbf{F}}_{\alpha}^{N} 
    -
    \left(
        \bar{\mathbf{D}}_{\alpha}^{NN}
    \right)^{-1} 
    \bar{\mathbf{D}}_{\alpha}^{NC} \bar{\mathbf{u}}^{C} 
\label{Internal Freedom Schur 1} 
\end{equation}


\begin{equation}
    \bar{\mathbf{u}}_{\beta}^{N} 
    =
    \left(
        \bar{\mathbf{D}}_{\beta}^{NN}
    \right)^{-1} 
    \bar{\mathbf{F}}_{\beta}^{N} 
    -
    \left(
        \bar{\mathbf{D}}_{\beta}^{NN}
    \right)^{-1} 
    \bar{\mathbf{D}}_{\beta}^{NC} \bar{\mathbf{u}}^{C} 
\label{Internal Freedom Schur 2} 
\end{equation}

Equations \eqref{Internal Freedom Schur 1} and \eqref{Internal Freedom Schur 2} are substituted into the formula of constraint mode's force from equation \eqref{Modal SoE} to form the following equation, with each terms detailed in subsequent equations. 

\begin{equation}
    \left[
        \mathbf{S}_{\alpha} + 
        \mathbf{S}_{\beta} 
    \right]
    \bar{\mathbf{u}}^{C} 
    = 
    \mathbf{P}_{\alpha} + 
    \mathbf{P}_{\beta}
\end{equation}

\begin{equation}
    \mathbf{S}_{\alpha} 
    =
    \bar{\mathbf{D}}_{\alpha}^{CC} - 
    \bar{\mathbf{D}}_{\alpha}^{CN} 
    \left(
        \bar{\mathbf{D}}_{\alpha}^{NN} 
    \right)^{-1} 
    \bar{\mathbf{D}}_{\alpha}^{NC}
\end{equation}

\begin{equation}
    \mathbf{S}_{\beta} 
    =
    \bar{\mathbf{D}}_{\beta}^{CC} - 
    \bar{\mathbf{D}}_{\beta}^{CN} 
    \left(
        \bar{\mathbf{D}}_{\beta}^{NN} 
    \right)^{-1} 
    \bar{\mathbf{D}}_{\beta}^{NC}
\end{equation}

\begin{equation}
    \mathbf{P}_{\alpha} 
    =
    \bar{\mathbf{F}}_{\alpha}^{C} - 
    \bar{\mathbf{D}}_{\alpha}^{CN} 
    \left(
        \bar{\mathbf{D}}_{\alpha}^{NN} 
    \right)^{-1} 
    \bar{\mathbf{F}}_{\alpha}^{N} 
\end{equation}

\begin{equation}
    \mathbf{P}_{\beta} 
    =
    \bar{\mathbf{F}}_{\beta}^{C} - 
    \bar{\mathbf{D}}_{\beta}^{CN} 
    \left(
        \bar{\mathbf{D}}_{\beta}^{NN} 
    \right)^{-1} 
    \bar{\mathbf{F}}_{\beta}^{N} 
\end{equation}

The procedure is parallelised by assigning each substructure to a different computing resource and compute the Schur complement matrix $\mathbf{S}_{\cdot}$ and force $\mathbf{P}_{\cdot}$ independently.
After solving for the constraint modes, the result is substituted back and the internal modes for each domain can be computed independently using the following two equations: 

\begin{equation}
    \bar{\mathbf{D}}_{\alpha}^{NN} 
    \bar{\mathbf{u}}_{\alpha}^{N} 
    =
    \bar{\mathbf{F}}_{\alpha}^{N} - 
    \bar{\mathbf{D}}_{\alpha}^{NC} 
    \bar{\mathbf{u}}^{C} 
\end{equation}

\begin{equation}
    \bar{\mathbf{D}}_{\beta}^{NN} 
    \bar{\mathbf{u}}_{\beta}^{N} 
    =
    \bar{\mathbf{F}}_{\beta}^{N} - 
    \bar{\mathbf{D}}_{\beta}^{NC} 
    \bar{\mathbf{u}}^{C} 
\end{equation}

\vspace{12pt} 
The generalisation for more than two substructures is adapted from \cite{chatterjee2020uncertainty}. 
\vspace{12pt} 

While all constraint DoFs belong to two substructures before, this is not the case for more than two substructures. 
However, by collecting all constraint DoFs into the vector $\bar{\mathbf{u}}^{C}$, the formulation for arbitrary number of substructures is straightforward and given in the following. 

\begin{equation}
    \left[
        \sum_{r=1}^{n_{ss}}{
            \mathbf{S}_{r} 
        }
    \right]
    \bar{\mathbf{u}}^{C} 
    = 
    \sum_{r=1}^{n_{ss}}{
        \mathbf{P}_{r} 
    }
\end{equation}

Just like before, each substructure is assigned to a different computating resource and the Schur complement matrix and force are computed independently, so is the internal DoFs upon substitution of constraint DoFs. 

\end{document}