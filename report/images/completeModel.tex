% -----------------------------------------------------------------------------
% Master thesis in the study program computational mechanics
%
% B.Sc. Rezha Adrian Tanuharja - 03751261
% M.Sc. Felix Schneider (supervisor)
%
% images/completeModel.tex
% Last edited 03 November 2023
% -----------------------------------------------------------------------------

\begin{figure}[H]
    \centering

    \begin{tikzpicture}[
        scale=0.35, 
        %
        % Isometric-esque appearance
        x={( 0.6cm,-0.2cm)}, 
        y={( 0.5cm, 0.3cm)}, 
        z={( 0.0cm, 1.0cm)}, 
        %
        draw opacity=0.7
    ]

        % Define grid parameters
        % maxes are lengths - one step because of plotting
        \def\xmin{0} \def\xmax{47.0}
        \def\ymin{0} \def\ymax{22.8}
        % the dimension of each rectangular element
        \def\xstep{1.0}
        \def\ystep{1.2}

        % Define clamping wall parameters
        \def\wallh{1.2}
        \def\wallt{1.0}

        % Define markers' diameter
        \def\noder{0.5}

        % Draw the side left clamping wall 
        \draw[pattern=north west lines] 
            (   \xmin, \ymin, -\wallh) -- 
            (   \xmin, \ymin,  \wallh) -- 
            ( -\wallt, \ymin,  \wallh) -- 
            ( -\wallt, \ymin, -\wallh) -- 
            cycle;

        % Draw the top left clamping wall
        \draw[pattern=north west lines] 
            (   \xmin, \ymin       ,  \wallh) -- 
            (   \xmin, \ymax+\ystep,  \wallh) -- 
            ( -\wallt, \ymax+\ystep,  \wallh) -- 
            ( -\wallt, \ymin       ,  \wallh) -- 
            cycle;

        % Draw the front left clamping wall
        \draw[fill=white]
            (   \xmin, \ymin       , -\wallh) -- 
            (   \xmin, \ymax+\ystep, -\wallh) --
            (   \xmin, \ymax+\ystep,  \wallh) --
            (   \xmin, \ymin       ,  \wallh) -- 
            cycle; 
    
        % Draw grid
        \foreach \x in {\xmin,\xstep,...,\xmax}
            \foreach \y in {\ymin,\ystep,...,\ymax}
                \draw[fill=white] 
                      ( \x   , \y   , 0) -- 
                    ++( \xstep, 0    , 0) -- 
                    ++( 0    , \ystep, 0) -- 
                    ++(-\xstep, 0    , 0) -- 
                    cycle;

        % Put the back right clamping wall in front of the grid
        \draw[fill=white]
            (   \xmax+\wallt+\xstep, \ymin       , -\wallh) -- 
            (   \xmax+\wallt+\xstep, \ymax+\ystep, -\wallh) --
            (   \xmax+\wallt+\xstep, \ymax+\ystep,  \wallh) --
            (   \xmax+\wallt+\xstep, \ymin       ,  \wallh) -- 
            cycle; 

        % Draw the back right clamping wall
        \draw[pattern=north east lines]
            (   \xmax+\wallt+\xstep, \ymin       , -\wallh) -- 
            (   \xmax+\wallt+\xstep, \ymax+\ystep, -\wallh) --
            (   \xmax+\wallt+\xstep, \ymax+\ystep,  \wallh) --
            (   \xmax+\wallt+\xstep, \ymin       ,  \wallh) -- 
            cycle; 

        % Put the side right clamping wall in front of the grid
        \draw[fill=white] 
            ( \xmax+\xstep       , \ymin, -\wallh) -- 
            ( \xmax+\xstep       , \ymin,  \wallh) -- 
            ( \xmax+\xstep+\wallt, \ymin,  \wallh) -- 
            ( \xmax+\xstep+\wallt, \ymin, -\wallh) -- 
            cycle;
        
        % Draw the side right clamping wall
        \draw[pattern=north west lines] 
            ( \xmax+\xstep       , \ymin, -\wallh) -- 
            ( \xmax+\xstep       , \ymin,  \wallh) -- 
            ( \xmax+\xstep+\wallt, \ymin,  \wallh) -- 
            ( \xmax+\xstep+\wallt, \ymin, -\wallh) -- 
            cycle;
        
        % Put the top right clamping wall in front of the grid
        \draw[fill=white] 
            ( \xmax+\xstep       , \ymin       ,  \wallh) -- 
            ( \xmax+\xstep       , \ymax+\ystep,  \wallh) -- 
            ( \xmax+\xstep+\wallt, \ymax+\ystep,  \wallh) -- 
            ( \xmax+\xstep+\wallt, \ymin       ,  \wallh) -- 
            cycle;

        % Draw the top right clamping wall
        \draw[pattern=north west lines] 
            ( \xmax+\xstep       , \ymin       ,  \wallh) -- 
            ( \xmax+\xstep       , \ymax+\ystep,  \wallh) -- 
            ( \xmax+\xstep+\wallt, \ymax+\ystep,  \wallh) -- 
            ( \xmax+\xstep+\wallt, \ymin       ,  \wallh) -- 
            cycle;

        % Draw a marker for the node B
        \draw[fill=red] 
            ( 8*\xstep, 9.5*\ystep ) ellipse (0.3 and 0.3);
        % Draw a line pointing at the marker for the node B
        \draw[thick]
            (  8*\xstep, 9.5*\ystep,        0 ) --
            ( -1*\xstep, 9.5*\ystep, 2*\wallh );
        % Add label B
        \node at
            ( -2*\xstep, 9.5*\ystep, 2*\wallh+0.6*\xstep ) {B};

        % Draw a marker for the node A
        \draw[fill=red] 
            ( 32*\xstep, 20*\ystep ) ellipse (0.3 and 0.3);
        % Draw a line pointing at the two markers for the node A
        \draw[thick]
            ( 32*\xstep         , 20*\ystep       , 0               ) --
            ( 32*\xstep+7*\xstep, 20*\ystep+\ystep, \wallh+3*\xstep );
        % Add label A
        \node at (32*\xstep+8*\xstep,20*\ystep+\ystep, \wallh+4*\xstep) {A};

    \end{tikzpicture}
    \caption{Meshed Complete Plate Model with Locations of Nodes A and B}
    \label{fig: complete model}
\end{figure}