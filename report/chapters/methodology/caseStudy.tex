% -----------------------------------------------------------------------------
% Master thesis in the study program computational mechanics
%
% B.Sc. Rezha Adrian Tanuharja - 03751261
% M.Sc. Felix Schneider (supervisor)
%
% chapters/methodology/caseStudy.tex
% Last edited 03 November 2023
% -----------------------------------------------------------------------------

\section{Case Study}
\label{sec: case study}

This study demonstrates the practical application of the new framework using a simple yet relevant example: a plate clamped at both of its ends.
The choice of case is deliberate, as it is sufficiently large for the substructuring methods while still maintaining the feasibility of direct MC simulations.
Figure \ref{fig: case study} illustrates the plate and its dimensions.

% -----------------------------------------------------------------------------
% Master thesis in the study program computational mechanics
%
% B.Sc. Rezha Adrian Tanuharja - 03751261
% M.Sc. Felix Schneider (supervisor)
%
% chapters/methodology/caseStudy.tex
% Last edited 03 November 2023
% -----------------------------------------------------------------------------

\section{Case Study}
\label{sec: case study}

This study demonstrates the practical application of the new framework using a simple yet relevant example: a plate clamped at both of its ends.
The choice of case is deliberate, as it is sufficiently large for the substructuring methods while still maintaining the feasibility of direct MC simulations.
Figure \ref{fig: case study} illustrates the plate and its dimensions.

% -----------------------------------------------------------------------------
% Master thesis in the study program computational mechanics
%
% B.Sc. Rezha Adrian Tanuharja - 03751261
% M.Sc. Felix Schneider (supervisor)
%
% chapters/methodology/caseStudy.tex
% Last edited 03 November 2023
% -----------------------------------------------------------------------------

\section{Case Study}
\label{sec: case study}

This study demonstrates the practical application of the new framework using a simple yet relevant example: a plate clamped at both of its ends.
The choice of case is deliberate, as it is sufficiently large for the substructuring methods while still maintaining the feasibility of direct MC simulations.
Figure \ref{fig: case study} illustrates the plate and its dimensions.

% -----------------------------------------------------------------------------
% Master thesis in the study program computational mechanics
%
% B.Sc. Rezha Adrian Tanuharja - 03751261
% M.Sc. Felix Schneider (supervisor)
%
% chapters/methodology/caseStudy.tex
% Last edited 03 November 2023
% -----------------------------------------------------------------------------

\section{Case Study}
\label{sec: case study}

This study demonstrates the practical application of the new framework using a simple yet relevant example: a plate clamped at both of its ends.
The choice of case is deliberate, as it is sufficiently large for the substructuring methods while still maintaining the feasibility of direct MC simulations.
Figure \ref{fig: case study} illustrates the plate and its dimensions.

\input{images/caseStudy}

Table \ref{table: Plate Parameter} lists the relevant plate's parameters.
The plate's density and thickness are uncertain and modeled by random variables.
The parameters $\xi_{1}$ and $\xi_{2}$ are statistically independent standard normal random variables.

\begin{table}[H]
    \setlength{\extrarowheight}{2pt}
    \centering
    \begin{tabular}{
        L{0.32\textwidth}rccC{0.25\textwidth}
    }
        \hline
        Parameter & \multicolumn{3}{c}{Value} & Unit \\
        \hline
        Thickness &
            $(1.0+0.2\xi_{1})$ & $\times$ & $0.01$ &
            \einheit{m} \\
        Density & 
            $(1.0+0.2\xi_{2})$ & $\times$ & $7850.0$ &
            \einheit{kg/{m}^{3}} \\
        Shear modulus & 
            \multicolumn{3}{c}{$79.3$} &
            \einheit{GPa} \\
        Shear correction factor & 
            \multicolumn{3}{c}{$0.83$} &
            -- \\
        Poisson ratio & 
            \multicolumn{3}{c}{$0.3$} &
            -- \\
        \hline
    \end{tabular}
    \caption{Plate's Parameters in The Case Study}
    \label{table: Plate Parameter}
\end{table}

Two discrete models approximate the plate using the FE method with Kirchoff plate theory: the complete plate model and the substructured plate model.
The proposed framework uses the latter to generate training data for the NI-RPCE models, which will approximate the complete plate model.
On the other hand, the FRFs from direct MCS using the complete plate model serve as the baselines for comparison.

\input{chapters/methodology/caseStudy/completeModel}
\input{chapters/methodology/caseStudy/substructuredModel}
\input{chapters/methodology/caseStudy/surrogateModel}

Table \ref{table: Plate Parameter} lists the relevant plate's parameters.
The plate's density and thickness are uncertain and modeled by random variables.
The parameters $\xi_{1}$ and $\xi_{2}$ are statistically independent standard normal random variables.

\begin{table}[H]
    \setlength{\extrarowheight}{2pt}
    \centering
    \begin{tabular}{
        L{0.32\textwidth}rccC{0.25\textwidth}
    }
        \hline
        Parameter & \multicolumn{3}{c}{Value} & Unit \\
        \hline
        Thickness &
            $(1.0+0.2\xi_{1})$ & $\times$ & $0.01$ &
            \einheit{m} \\
        Density & 
            $(1.0+0.2\xi_{2})$ & $\times$ & $7850.0$ &
            \einheit{kg/{m}^{3}} \\
        Shear modulus & 
            \multicolumn{3}{c}{$79.3$} &
            \einheit{GPa} \\
        Shear correction factor & 
            \multicolumn{3}{c}{$0.83$} &
            -- \\
        Poisson ratio & 
            \multicolumn{3}{c}{$0.3$} &
            -- \\
        \hline
    \end{tabular}
    \caption{Plate's Parameters in The Case Study}
    \label{table: Plate Parameter}
\end{table}

Two discrete models approximate the plate using the FE method with Kirchoff plate theory: the complete plate model and the substructured plate model.
The proposed framework uses the latter to generate training data for the NI-RPCE models, which will approximate the complete plate model.
On the other hand, the FRFs from direct MCS using the complete plate model serve as the baselines for comparison.

% -----------------------------------------------------------------------------
% Master thesis in the study program computational mechanics
%
% B.Sc. Rezha Adrian Tanuharja - 03751261
% M.Sc. Felix Schneider (supervisor)
%
% chapters/methodology/caseStudy/completeModel.tex
% Last edited 03 November 2023
% -----------------------------------------------------------------------------

\subsection{Complete Plate Model}
\label{ssec: full model}

The discretization process uses biquadratic rectangular elements in a uniform mesh. There are three DOFs at each node: the vertical displacement, the rotation about the $x$ axis, and the rotation about the $y$ axis.
Tabel \ref{table: Mesh Parameter} provides the mesh's parameters.

\begin{table}[H]
    \setlength{\extrarowheight}{2pt}
    \centering 
    \begin{tabular}{
        L{0.3\textwidth} C{0.2\textwidth}
    }
        \hline
        Parameter & 
        Value \\
        \hline
        Number of elements (x) &
        48 \\
        Number of elements (y) &
        20 \\
        & \\
        Total number of nodes &
        3977 \\
        Total number of DOFs &
        11931 \\
        \hline 
    \end{tabular}
    \caption{Complete Plate Model Mesh's Parameters}
    \label{table: Mesh Parameter}
\end{table}%
%
The case study focuses on two particular nodes: node A located on the edge of the plate and node B located inside the plate's interior.
Table \ref{table: Node Coordinates} provides the coordinates of these two nodes.

\begin{table}[H]
    \setlength{\extrarowheight}{2pt}
    \centering
    \begin{tabular}{
        C{0.15\textwidth} C{0.15\textwidth} C{0.15\textwidth}
    }
        \hline
        Node & $x$ [m] & $y$ [m] \\
        \hline
        A & $5.\bar{3}$ & $4.0$ \\
        B & $1.\bar{3}$ & $1.9$ \\
        \hline
    \end{tabular}
    \caption{Coordinates of Two Nodes of Interest}
    \label{table: Node Coordinates}
\end{table}

In the evaluation, the author applies a unit vertical force on node A and computes the vertical accelerations of nodes A and B.
Figure \ref{fig: complete model} illustrates the meshed full plate model and the two nodes.

\input{images/completeModel}
% -----------------------------------------------------------------------------
% Master thesis in the study program computational mechanics
%
% B.Sc. Rezha Adrian Tanuharja - 03751261
% M.Sc. Felix Schneider (supervisor)
%
% images/substructuredModel.tex
% Last edited 03 November 2023
% -----------------------------------------------------------------------------

\begin{figure}[H]
    \centering

    \begin{tikzpicture}[
        scale=0.35, 
        %
        % Isometric-esque appearance
        x={( 0.6cm,-0.2cm)}, 
        y={( 0.5cm, 0.3cm)}, 
        z={( 0.0cm, 1.0cm)}, 
        %
        draw opacity=0.7
    ]

        % Define the gap between the components
        \def\gap{2.0}

        % Define grid parameters
        % maxes are lengths - one step because of plotting
        \def\xmin{0.0} \def\xmax{15.0}
        \def\ymin{0.0} \def\ymax{22.8}

        % the dimension of each rectangular element
        \def\xstep{1.0}
        \def\ystep{1.2}

        % Define clamping wall parameters
        \def\wallh{1.2}
        \def\wallt{1.0}

        % Define markers' diameter
        \def\noder{0.5}

        % Draw the side left clamping wall 
        \draw[pattern=north west lines] 
            (   \xmin, \ymin, -\wallh) -- 
            (   \xmin, \ymin,  \wallh) -- 
            ( -\wallt, \ymin,  \wallh) -- 
            ( -\wallt, \ymin, -\wallh) -- 
            cycle;

        % Draw the top left clamping wall
        \draw[pattern=north west lines] 
            (   \xmin, \ymin       ,  \wallh) -- 
            (   \xmin, \ymax+\ystep,  \wallh) -- 
            ( -\wallt, \ymax+\ystep,  \wallh) -- 
            ( -\wallt, \ymin       ,  \wallh) -- 
            cycle;

        % Draw the front left clamping wall
        \draw[fill=white]
            (   \xmin, \ymin       , -\wallh) -- 
            (   \xmin, \ymax+\ystep, -\wallh) --
            (   \xmin, \ymax+\ystep,  \wallh) --
            (   \xmin, \ymin       ,  \wallh) -- 
            cycle; 
    
        % Draw the left component
        \foreach \x in {\xmin,\xstep,...,\xmax}
            \foreach \y in {\ymin,\ystep,...,\ymax}
                \draw[fill=white] 
                      ( \x   , \y   , 0) -- 
                    ++( \xstep, 0    , 0) -- 
                    ++( 0    , \ystep, 0) -- 
                    ++(-\xstep, 0    , 0) -- 
                    cycle;

        \def\xmax{13.0}
        \def\offset{16.0}
        \foreach \x in {\xmin,\xstep,...,\xmax}
            \foreach \y in {\ymin,\ystep,...,\ymax}
                \draw[fill=white] 
                      ( \x + \offset + \gap  , \y, 0) -- 
                    ++( \xstep, 0    , 0) -- 
                    ++( 0    , \ystep, 0) -- 
                    ++(-\xstep, 0    , 0) -- 
                    cycle;

        \def\xmax{17.0}
        \def\offset{30.0}
        \foreach \x in {\xmin,\xstep,...,\xmax}
            \foreach \y in {\ymin,\ystep,...,\ymax}
                \draw[fill=white] 
                      ( \x + \offset + 2*\gap  , \y, 0) -- 
                    ++( \xstep, 0    , 0) -- 
                    ++( 0    , \ystep, 0) -- 
                    ++(-\xstep, 0    , 0) -- 
                    cycle;

        % Put the back right clamping wall in front of the grid
        \def\offset{48.0}
        \draw[fill=white]
            ( \offset+\wallt+2*\gap, \ymin       , -\wallh) -- 
            ( \offset+\wallt+2*\gap, \ymax+\ystep, -\wallh) --
            ( \offset+\wallt+2*\gap, \ymax+\ystep,  \wallh) --
            ( \offset+\wallt+2*\gap, \ymin       ,  \wallh) -- 
            cycle; 

        % Draw the back right clamping wall
        \draw[pattern=north east lines]
            ( \offset+\wallt+2*\gap, \ymin       , -\wallh) -- 
            ( \offset+\wallt+2*\gap, \ymax+\ystep, -\wallh) --
            ( \offset+\wallt+2*\gap, \ymax+\ystep,  \wallh) --
            ( \offset+\wallt+2*\gap, \ymin       ,  \wallh) -- 
            cycle; 

        % Put the side right clamping wall in front of the grid
        \draw[fill=white] 
            ( \offset+2*\gap       , \ymin, -\wallh) -- 
            ( \offset+2*\gap       , \ymin,  \wallh) -- 
            ( \offset+2*\gap+\wallt, \ymin,  \wallh) -- 
            ( \offset+2*\gap+\wallt, \ymin, -\wallh) -- 
            cycle;
        
        % Draw the side right clamping wall
        \draw[pattern=north west lines] 
            ( \offset+2*\gap       , \ymin, -\wallh) -- 
            ( \offset+2*\gap       , \ymin,  \wallh) -- 
            ( \offset+2*\gap+\wallt, \ymin,  \wallh) -- 
            ( \offset+2*\gap+\wallt, \ymin, -\wallh) -- 
            cycle;
        
        % Put the top right clamping wall in front of the grid
        \draw[fill=white] 
            ( \offset+2*\gap       , \ymin       ,  \wallh) -- 
            ( \offset+2*\gap       , \ymax+\ystep,  \wallh) -- 
            ( \offset+2*\gap+\wallt, \ymax+\ystep,  \wallh) -- 
            ( \offset+2*\gap+\wallt, \ymin       ,  \wallh) -- 
            cycle;

        % Draw the top right clamping wall
        \draw[pattern=north west lines] 
            ( \offset+2*\gap       , \ymin       ,  \wallh) -- 
            ( \offset+2*\gap       , \ymax+\ystep,  \wallh) -- 
            ( \offset+2*\gap+\wallt, \ymax+\ystep,  \wallh) -- 
            ( \offset+2*\gap+\wallt, \ymin       ,  \wallh) -- 
            cycle;

        % Draw a marker for the node B
        \draw[fill=red] 
            ( 8*\xstep, 9.5*\ystep ) ellipse (0.3 and 0.3);
        % Draw a line pointing at the marker for the node B
        \draw[thick]
            (  8*\xstep, 9.5*\ystep,        0 ) --
            ( -1*\xstep, 9.5*\ystep, 2*\wallh );
        % Add label B
        \node at
            ( -2*\xstep, 9.5*\ystep, 2*\wallh+0.6*\xstep ) {B};

        % Draw a marker for the node A
        \draw[fill=red] 
            ( 32*\xstep+2*\gap, 20*\ystep ) ellipse (0.3 and 0.3);
        % Draw a line pointing at the two markers for the node A
        \draw[thick]
            ( 32*\xstep+2*\gap         , 20*\ystep       , 0               ) --
            ( 32*\xstep+  \gap+7*\xstep, 20*\ystep+\ystep, \wallh+3*\xstep );
        % Add label A
        \node at (32*\xstep+8*\xstep+\gap,20*\ystep+\ystep, \wallh+4*\xstep) {A};

    \end{tikzpicture}
    \caption{Meshed Substructured Plate Model with Locations of Nodes A and B}
    \label{fig: substructured model}
\end{figure}
% -----------------------------------------------------------------------------
% Master thesis in the study program computational mechanics
%
% B.Sc. Rezha Adrian Tanuharja - 03751261
% M.Sc. Felix Schneider (supervisor)
%
% chapters/methodology/caseStudy/surrogateModel.tex
% Last edited 03 November 2023
% -----------------------------------------------------------------------------

\subsection{Sparse NI-RPCE Model}
\label{ssec: surrogate model}

This study uses RPCE models with products of probabilist Hermite polynomials as the basis functions, as in \eqref{prob Hermite products}.
The numerator is a PCE with an order of $5$ while the denominator is a PCE with an order of $6$.
Therefore, the basis functions in the numerator and denominator satisfy
\begin{equation}
    \Psi_{k}\left(\mathbf{\Xi}\right)
    =
    {He}_{r_{k}} \left(\xi_{1}\right)
    \cdot
    {He}_{s_{k}} \left(\xi_{2}\right)
    \phantom{x}
    \forall 
    \phantom{x}
    r_{k}, s_{k} \in \mathbb{N}, 
    \phantom{x}
    r_{k} + s_{k} \leq 5
\end{equation}
and
\begin{equation}
    \Psi_{l}\left(\mathbf{\Xi}\right)
    =
    {He}_{r_{l}} \left(\xi_{1}\right)
    \cdot
    {He}_{s_{l}} \left(\xi_{2}\right)
    \phantom{x}
    \forall 
    \phantom{x}
    r_{l}, s_{l} \in \mathbb{N}, 
    \phantom{x}
    r_{l} + s_{l} \leq 6,
\end{equation}
respectively.
The author obtains the coefficient in \eqref{RPCE_approx} by solving \eqref{SVD problem}.
There are $49$ basis functions in the numerator and denominator of the RPCE models.
The author reduces the number of basis functions by following the algorithms \ref{alg: sparse RPCE numerator} and \ref{alg: sparse RPCE denominator} with $\epsilon_{0}=1.0$.

Table \ref{table: Plate Parameter} lists the relevant plate's parameters.
The plate's density and thickness are uncertain and modeled by random variables.
The parameters $\xi_{1}$ and $\xi_{2}$ are statistically independent standard normal random variables.

\begin{table}[H]
    \setlength{\extrarowheight}{2pt}
    \centering
    \begin{tabular}{
        L{0.32\textwidth}rccC{0.25\textwidth}
    }
        \hline
        Parameter & \multicolumn{3}{c}{Value} & Unit \\
        \hline
        Thickness &
            $(1.0+0.2\xi_{1})$ & $\times$ & $0.01$ &
            \einheit{m} \\
        Density & 
            $(1.0+0.2\xi_{2})$ & $\times$ & $7850.0$ &
            \einheit{kg/{m}^{3}} \\
        Shear modulus & 
            \multicolumn{3}{c}{$79.3$} &
            \einheit{GPa} \\
        Shear correction factor & 
            \multicolumn{3}{c}{$0.83$} &
            -- \\
        Poisson ratio & 
            \multicolumn{3}{c}{$0.3$} &
            -- \\
        \hline
    \end{tabular}
    \caption{Plate's Parameters in The Case Study}
    \label{table: Plate Parameter}
\end{table}

Two discrete models approximate the plate using the FE method with Kirchoff plate theory: the complete plate model and the substructured plate model.
The proposed framework uses the latter to generate training data for the NI-RPCE models, which will approximate the complete plate model.
On the other hand, the FRFs from direct MCS using the complete plate model serve as the baselines for comparison.

% -----------------------------------------------------------------------------
% Master thesis in the study program computational mechanics
%
% B.Sc. Rezha Adrian Tanuharja - 03751261
% M.Sc. Felix Schneider (supervisor)
%
% chapters/methodology/caseStudy/completeModel.tex
% Last edited 03 November 2023
% -----------------------------------------------------------------------------

\subsection{Complete Plate Model}
\label{ssec: full model}

The discretization process uses biquadratic rectangular elements in a uniform mesh. There are three DOFs at each node: the vertical displacement, the rotation about the $x$ axis, and the rotation about the $y$ axis.
Tabel \ref{table: Mesh Parameter} provides the mesh's parameters.

\begin{table}[H]
    \setlength{\extrarowheight}{2pt}
    \centering 
    \begin{tabular}{
        L{0.3\textwidth} C{0.2\textwidth}
    }
        \hline
        Parameter & 
        Value \\
        \hline
        Number of elements (x) &
        48 \\
        Number of elements (y) &
        20 \\
        & \\
        Total number of nodes &
        3977 \\
        Total number of DOFs &
        11931 \\
        \hline 
    \end{tabular}
    \caption{Complete Plate Model Mesh's Parameters}
    \label{table: Mesh Parameter}
\end{table}%
%
The case study focuses on two particular nodes: node A located on the edge of the plate and node B located inside the plate's interior.
Table \ref{table: Node Coordinates} provides the coordinates of these two nodes.

\begin{table}[H]
    \setlength{\extrarowheight}{2pt}
    \centering
    \begin{tabular}{
        C{0.15\textwidth} C{0.15\textwidth} C{0.15\textwidth}
    }
        \hline
        Node & $x$ [m] & $y$ [m] \\
        \hline
        A & $5.\bar{3}$ & $4.0$ \\
        B & $1.\bar{3}$ & $1.9$ \\
        \hline
    \end{tabular}
    \caption{Coordinates of Two Nodes of Interest}
    \label{table: Node Coordinates}
\end{table}

In the evaluation, the author applies a unit vertical force on node A and computes the vertical accelerations of nodes A and B.
Figure \ref{fig: complete model} illustrates the meshed full plate model and the two nodes.

% -----------------------------------------------------------------------------
% Master thesis in the study program computational mechanics
%
% B.Sc. Rezha Adrian Tanuharja - 03751261
% M.Sc. Felix Schneider (supervisor)
%
% chapters/methodology/caseStudy/completeModel.tex
% Last edited 03 November 2023
% -----------------------------------------------------------------------------

\subsection{Complete Plate Model}
\label{ssec: full model}

The discretization process uses biquadratic rectangular elements in a uniform mesh. There are three DOFs at each node: the vertical displacement, the rotation about the $x$ axis, and the rotation about the $y$ axis.
Tabel \ref{table: Mesh Parameter} provides the mesh's parameters.

\begin{table}[H]
    \setlength{\extrarowheight}{2pt}
    \centering 
    \begin{tabular}{
        L{0.3\textwidth} C{0.2\textwidth}
    }
        \hline
        Parameter & 
        Value \\
        \hline
        Number of elements (x) &
        48 \\
        Number of elements (y) &
        20 \\
        & \\
        Total number of nodes &
        3977 \\
        Total number of DOFs &
        11931 \\
        \hline 
    \end{tabular}
    \caption{Complete Plate Model Mesh's Parameters}
    \label{table: Mesh Parameter}
\end{table}%
%
The case study focuses on two particular nodes: node A located on the edge of the plate and node B located inside the plate's interior.
Table \ref{table: Node Coordinates} provides the coordinates of these two nodes.

\begin{table}[H]
    \setlength{\extrarowheight}{2pt}
    \centering
    \begin{tabular}{
        C{0.15\textwidth} C{0.15\textwidth} C{0.15\textwidth}
    }
        \hline
        Node & $x$ [m] & $y$ [m] \\
        \hline
        A & $5.\bar{3}$ & $4.0$ \\
        B & $1.\bar{3}$ & $1.9$ \\
        \hline
    \end{tabular}
    \caption{Coordinates of Two Nodes of Interest}
    \label{table: Node Coordinates}
\end{table}

In the evaluation, the author applies a unit vertical force on node A and computes the vertical accelerations of nodes A and B.
Figure \ref{fig: complete model} illustrates the meshed full plate model and the two nodes.

\input{images/completeModel}
% -----------------------------------------------------------------------------
% Master thesis in the study program computational mechanics
%
% B.Sc. Rezha Adrian Tanuharja - 03751261
% M.Sc. Felix Schneider (supervisor)
%
% images/substructuredModel.tex
% Last edited 03 November 2023
% -----------------------------------------------------------------------------

\begin{figure}[H]
    \centering

    \begin{tikzpicture}[
        scale=0.35, 
        %
        % Isometric-esque appearance
        x={( 0.6cm,-0.2cm)}, 
        y={( 0.5cm, 0.3cm)}, 
        z={( 0.0cm, 1.0cm)}, 
        %
        draw opacity=0.7
    ]

        % Define the gap between the components
        \def\gap{2.0}

        % Define grid parameters
        % maxes are lengths - one step because of plotting
        \def\xmin{0.0} \def\xmax{15.0}
        \def\ymin{0.0} \def\ymax{22.8}

        % the dimension of each rectangular element
        \def\xstep{1.0}
        \def\ystep{1.2}

        % Define clamping wall parameters
        \def\wallh{1.2}
        \def\wallt{1.0}

        % Define markers' diameter
        \def\noder{0.5}

        % Draw the side left clamping wall 
        \draw[pattern=north west lines] 
            (   \xmin, \ymin, -\wallh) -- 
            (   \xmin, \ymin,  \wallh) -- 
            ( -\wallt, \ymin,  \wallh) -- 
            ( -\wallt, \ymin, -\wallh) -- 
            cycle;

        % Draw the top left clamping wall
        \draw[pattern=north west lines] 
            (   \xmin, \ymin       ,  \wallh) -- 
            (   \xmin, \ymax+\ystep,  \wallh) -- 
            ( -\wallt, \ymax+\ystep,  \wallh) -- 
            ( -\wallt, \ymin       ,  \wallh) -- 
            cycle;

        % Draw the front left clamping wall
        \draw[fill=white]
            (   \xmin, \ymin       , -\wallh) -- 
            (   \xmin, \ymax+\ystep, -\wallh) --
            (   \xmin, \ymax+\ystep,  \wallh) --
            (   \xmin, \ymin       ,  \wallh) -- 
            cycle; 
    
        % Draw the left component
        \foreach \x in {\xmin,\xstep,...,\xmax}
            \foreach \y in {\ymin,\ystep,...,\ymax}
                \draw[fill=white] 
                      ( \x   , \y   , 0) -- 
                    ++( \xstep, 0    , 0) -- 
                    ++( 0    , \ystep, 0) -- 
                    ++(-\xstep, 0    , 0) -- 
                    cycle;

        \def\xmax{13.0}
        \def\offset{16.0}
        \foreach \x in {\xmin,\xstep,...,\xmax}
            \foreach \y in {\ymin,\ystep,...,\ymax}
                \draw[fill=white] 
                      ( \x + \offset + \gap  , \y, 0) -- 
                    ++( \xstep, 0    , 0) -- 
                    ++( 0    , \ystep, 0) -- 
                    ++(-\xstep, 0    , 0) -- 
                    cycle;

        \def\xmax{17.0}
        \def\offset{30.0}
        \foreach \x in {\xmin,\xstep,...,\xmax}
            \foreach \y in {\ymin,\ystep,...,\ymax}
                \draw[fill=white] 
                      ( \x + \offset + 2*\gap  , \y, 0) -- 
                    ++( \xstep, 0    , 0) -- 
                    ++( 0    , \ystep, 0) -- 
                    ++(-\xstep, 0    , 0) -- 
                    cycle;

        % Put the back right clamping wall in front of the grid
        \def\offset{48.0}
        \draw[fill=white]
            ( \offset+\wallt+2*\gap, \ymin       , -\wallh) -- 
            ( \offset+\wallt+2*\gap, \ymax+\ystep, -\wallh) --
            ( \offset+\wallt+2*\gap, \ymax+\ystep,  \wallh) --
            ( \offset+\wallt+2*\gap, \ymin       ,  \wallh) -- 
            cycle; 

        % Draw the back right clamping wall
        \draw[pattern=north east lines]
            ( \offset+\wallt+2*\gap, \ymin       , -\wallh) -- 
            ( \offset+\wallt+2*\gap, \ymax+\ystep, -\wallh) --
            ( \offset+\wallt+2*\gap, \ymax+\ystep,  \wallh) --
            ( \offset+\wallt+2*\gap, \ymin       ,  \wallh) -- 
            cycle; 

        % Put the side right clamping wall in front of the grid
        \draw[fill=white] 
            ( \offset+2*\gap       , \ymin, -\wallh) -- 
            ( \offset+2*\gap       , \ymin,  \wallh) -- 
            ( \offset+2*\gap+\wallt, \ymin,  \wallh) -- 
            ( \offset+2*\gap+\wallt, \ymin, -\wallh) -- 
            cycle;
        
        % Draw the side right clamping wall
        \draw[pattern=north west lines] 
            ( \offset+2*\gap       , \ymin, -\wallh) -- 
            ( \offset+2*\gap       , \ymin,  \wallh) -- 
            ( \offset+2*\gap+\wallt, \ymin,  \wallh) -- 
            ( \offset+2*\gap+\wallt, \ymin, -\wallh) -- 
            cycle;
        
        % Put the top right clamping wall in front of the grid
        \draw[fill=white] 
            ( \offset+2*\gap       , \ymin       ,  \wallh) -- 
            ( \offset+2*\gap       , \ymax+\ystep,  \wallh) -- 
            ( \offset+2*\gap+\wallt, \ymax+\ystep,  \wallh) -- 
            ( \offset+2*\gap+\wallt, \ymin       ,  \wallh) -- 
            cycle;

        % Draw the top right clamping wall
        \draw[pattern=north west lines] 
            ( \offset+2*\gap       , \ymin       ,  \wallh) -- 
            ( \offset+2*\gap       , \ymax+\ystep,  \wallh) -- 
            ( \offset+2*\gap+\wallt, \ymax+\ystep,  \wallh) -- 
            ( \offset+2*\gap+\wallt, \ymin       ,  \wallh) -- 
            cycle;

        % Draw a marker for the node B
        \draw[fill=red] 
            ( 8*\xstep, 9.5*\ystep ) ellipse (0.3 and 0.3);
        % Draw a line pointing at the marker for the node B
        \draw[thick]
            (  8*\xstep, 9.5*\ystep,        0 ) --
            ( -1*\xstep, 9.5*\ystep, 2*\wallh );
        % Add label B
        \node at
            ( -2*\xstep, 9.5*\ystep, 2*\wallh+0.6*\xstep ) {B};

        % Draw a marker for the node A
        \draw[fill=red] 
            ( 32*\xstep+2*\gap, 20*\ystep ) ellipse (0.3 and 0.3);
        % Draw a line pointing at the two markers for the node A
        \draw[thick]
            ( 32*\xstep+2*\gap         , 20*\ystep       , 0               ) --
            ( 32*\xstep+  \gap+7*\xstep, 20*\ystep+\ystep, \wallh+3*\xstep );
        % Add label A
        \node at (32*\xstep+8*\xstep+\gap,20*\ystep+\ystep, \wallh+4*\xstep) {A};

    \end{tikzpicture}
    \caption{Meshed Substructured Plate Model with Locations of Nodes A and B}
    \label{fig: substructured model}
\end{figure}
% -----------------------------------------------------------------------------
% Master thesis in the study program computational mechanics
%
% B.Sc. Rezha Adrian Tanuharja - 03751261
% M.Sc. Felix Schneider (supervisor)
%
% chapters/methodology/caseStudy/surrogateModel.tex
% Last edited 03 November 2023
% -----------------------------------------------------------------------------

\subsection{Sparse NI-RPCE Model}
\label{ssec: surrogate model}

This study uses RPCE models with products of probabilist Hermite polynomials as the basis functions, as in \eqref{prob Hermite products}.
The numerator is a PCE with an order of $5$ while the denominator is a PCE with an order of $6$.
Therefore, the basis functions in the numerator and denominator satisfy
\begin{equation}
    \Psi_{k}\left(\mathbf{\Xi}\right)
    =
    {He}_{r_{k}} \left(\xi_{1}\right)
    \cdot
    {He}_{s_{k}} \left(\xi_{2}\right)
    \phantom{x}
    \forall 
    \phantom{x}
    r_{k}, s_{k} \in \mathbb{N}, 
    \phantom{x}
    r_{k} + s_{k} \leq 5
\end{equation}
and
\begin{equation}
    \Psi_{l}\left(\mathbf{\Xi}\right)
    =
    {He}_{r_{l}} \left(\xi_{1}\right)
    \cdot
    {He}_{s_{l}} \left(\xi_{2}\right)
    \phantom{x}
    \forall 
    \phantom{x}
    r_{l}, s_{l} \in \mathbb{N}, 
    \phantom{x}
    r_{l} + s_{l} \leq 6,
\end{equation}
respectively.
The author obtains the coefficient in \eqref{RPCE_approx} by solving \eqref{SVD problem}.
There are $49$ basis functions in the numerator and denominator of the RPCE models.
The author reduces the number of basis functions by following the algorithms \ref{alg: sparse RPCE numerator} and \ref{alg: sparse RPCE denominator} with $\epsilon_{0}=1.0$.

Table \ref{table: Plate Parameter} lists the relevant plate's parameters.
The plate's density and thickness are uncertain and modeled by random variables.
The parameters $\xi_{1}$ and $\xi_{2}$ are statistically independent standard normal random variables.

\begin{table}[H]
    \setlength{\extrarowheight}{2pt}
    \centering
    \begin{tabular}{
        L{0.32\textwidth}rccC{0.25\textwidth}
    }
        \hline
        Parameter & \multicolumn{3}{c}{Value} & Unit \\
        \hline
        Thickness &
            $(1.0+0.2\xi_{1})$ & $\times$ & $0.01$ &
            \einheit{m} \\
        Density & 
            $(1.0+0.2\xi_{2})$ & $\times$ & $7850.0$ &
            \einheit{kg/{m}^{3}} \\
        Shear modulus & 
            \multicolumn{3}{c}{$79.3$} &
            \einheit{GPa} \\
        Shear correction factor & 
            \multicolumn{3}{c}{$0.83$} &
            -- \\
        Poisson ratio & 
            \multicolumn{3}{c}{$0.3$} &
            -- \\
        \hline
    \end{tabular}
    \caption{Plate's Parameters in The Case Study}
    \label{table: Plate Parameter}
\end{table}

Two discrete models approximate the plate using the FE method with Kirchoff plate theory: the complete plate model and the substructured plate model.
The proposed framework uses the latter to generate training data for the NI-RPCE models, which will approximate the complete plate model.
On the other hand, the FRFs from direct MCS using the complete plate model serve as the baselines for comparison.

% -----------------------------------------------------------------------------
% Master thesis in the study program computational mechanics
%
% B.Sc. Rezha Adrian Tanuharja - 03751261
% M.Sc. Felix Schneider (supervisor)
%
% chapters/methodology/caseStudy/completeModel.tex
% Last edited 03 November 2023
% -----------------------------------------------------------------------------

\subsection{Complete Plate Model}
\label{ssec: full model}

The discretization process uses biquadratic rectangular elements in a uniform mesh. There are three DOFs at each node: the vertical displacement, the rotation about the $x$ axis, and the rotation about the $y$ axis.
Tabel \ref{table: Mesh Parameter} provides the mesh's parameters.

\begin{table}[H]
    \setlength{\extrarowheight}{2pt}
    \centering 
    \begin{tabular}{
        L{0.3\textwidth} C{0.2\textwidth}
    }
        \hline
        Parameter & 
        Value \\
        \hline
        Number of elements (x) &
        48 \\
        Number of elements (y) &
        20 \\
        & \\
        Total number of nodes &
        3977 \\
        Total number of DOFs &
        11931 \\
        \hline 
    \end{tabular}
    \caption{Complete Plate Model Mesh's Parameters}
    \label{table: Mesh Parameter}
\end{table}%
%
The case study focuses on two particular nodes: node A located on the edge of the plate and node B located inside the plate's interior.
Table \ref{table: Node Coordinates} provides the coordinates of these two nodes.

\begin{table}[H]
    \setlength{\extrarowheight}{2pt}
    \centering
    \begin{tabular}{
        C{0.15\textwidth} C{0.15\textwidth} C{0.15\textwidth}
    }
        \hline
        Node & $x$ [m] & $y$ [m] \\
        \hline
        A & $5.\bar{3}$ & $4.0$ \\
        B & $1.\bar{3}$ & $1.9$ \\
        \hline
    \end{tabular}
    \caption{Coordinates of Two Nodes of Interest}
    \label{table: Node Coordinates}
\end{table}

In the evaluation, the author applies a unit vertical force on node A and computes the vertical accelerations of nodes A and B.
Figure \ref{fig: complete model} illustrates the meshed full plate model and the two nodes.

% -----------------------------------------------------------------------------
% Master thesis in the study program computational mechanics
%
% B.Sc. Rezha Adrian Tanuharja - 03751261
% M.Sc. Felix Schneider (supervisor)
%
% chapters/methodology/caseStudy/completeModel.tex
% Last edited 03 November 2023
% -----------------------------------------------------------------------------

\subsection{Complete Plate Model}
\label{ssec: full model}

The discretization process uses biquadratic rectangular elements in a uniform mesh. There are three DOFs at each node: the vertical displacement, the rotation about the $x$ axis, and the rotation about the $y$ axis.
Tabel \ref{table: Mesh Parameter} provides the mesh's parameters.

\begin{table}[H]
    \setlength{\extrarowheight}{2pt}
    \centering 
    \begin{tabular}{
        L{0.3\textwidth} C{0.2\textwidth}
    }
        \hline
        Parameter & 
        Value \\
        \hline
        Number of elements (x) &
        48 \\
        Number of elements (y) &
        20 \\
        & \\
        Total number of nodes &
        3977 \\
        Total number of DOFs &
        11931 \\
        \hline 
    \end{tabular}
    \caption{Complete Plate Model Mesh's Parameters}
    \label{table: Mesh Parameter}
\end{table}%
%
The case study focuses on two particular nodes: node A located on the edge of the plate and node B located inside the plate's interior.
Table \ref{table: Node Coordinates} provides the coordinates of these two nodes.

\begin{table}[H]
    \setlength{\extrarowheight}{2pt}
    \centering
    \begin{tabular}{
        C{0.15\textwidth} C{0.15\textwidth} C{0.15\textwidth}
    }
        \hline
        Node & $x$ [m] & $y$ [m] \\
        \hline
        A & $5.\bar{3}$ & $4.0$ \\
        B & $1.\bar{3}$ & $1.9$ \\
        \hline
    \end{tabular}
    \caption{Coordinates of Two Nodes of Interest}
    \label{table: Node Coordinates}
\end{table}

In the evaluation, the author applies a unit vertical force on node A and computes the vertical accelerations of nodes A and B.
Figure \ref{fig: complete model} illustrates the meshed full plate model and the two nodes.

% -----------------------------------------------------------------------------
% Master thesis in the study program computational mechanics
%
% B.Sc. Rezha Adrian Tanuharja - 03751261
% M.Sc. Felix Schneider (supervisor)
%
% chapters/methodology/caseStudy/completeModel.tex
% Last edited 03 November 2023
% -----------------------------------------------------------------------------

\subsection{Complete Plate Model}
\label{ssec: full model}

The discretization process uses biquadratic rectangular elements in a uniform mesh. There are three DOFs at each node: the vertical displacement, the rotation about the $x$ axis, and the rotation about the $y$ axis.
Tabel \ref{table: Mesh Parameter} provides the mesh's parameters.

\begin{table}[H]
    \setlength{\extrarowheight}{2pt}
    \centering 
    \begin{tabular}{
        L{0.3\textwidth} C{0.2\textwidth}
    }
        \hline
        Parameter & 
        Value \\
        \hline
        Number of elements (x) &
        48 \\
        Number of elements (y) &
        20 \\
        & \\
        Total number of nodes &
        3977 \\
        Total number of DOFs &
        11931 \\
        \hline 
    \end{tabular}
    \caption{Complete Plate Model Mesh's Parameters}
    \label{table: Mesh Parameter}
\end{table}%
%
The case study focuses on two particular nodes: node A located on the edge of the plate and node B located inside the plate's interior.
Table \ref{table: Node Coordinates} provides the coordinates of these two nodes.

\begin{table}[H]
    \setlength{\extrarowheight}{2pt}
    \centering
    \begin{tabular}{
        C{0.15\textwidth} C{0.15\textwidth} C{0.15\textwidth}
    }
        \hline
        Node & $x$ [m] & $y$ [m] \\
        \hline
        A & $5.\bar{3}$ & $4.0$ \\
        B & $1.\bar{3}$ & $1.9$ \\
        \hline
    \end{tabular}
    \caption{Coordinates of Two Nodes of Interest}
    \label{table: Node Coordinates}
\end{table}

In the evaluation, the author applies a unit vertical force on node A and computes the vertical accelerations of nodes A and B.
Figure \ref{fig: complete model} illustrates the meshed full plate model and the two nodes.

\input{images/completeModel}
% -----------------------------------------------------------------------------
% Master thesis in the study program computational mechanics
%
% B.Sc. Rezha Adrian Tanuharja - 03751261
% M.Sc. Felix Schneider (supervisor)
%
% images/substructuredModel.tex
% Last edited 03 November 2023
% -----------------------------------------------------------------------------

\begin{figure}[H]
    \centering

    \begin{tikzpicture}[
        scale=0.35, 
        %
        % Isometric-esque appearance
        x={( 0.6cm,-0.2cm)}, 
        y={( 0.5cm, 0.3cm)}, 
        z={( 0.0cm, 1.0cm)}, 
        %
        draw opacity=0.7
    ]

        % Define the gap between the components
        \def\gap{2.0}

        % Define grid parameters
        % maxes are lengths - one step because of plotting
        \def\xmin{0.0} \def\xmax{15.0}
        \def\ymin{0.0} \def\ymax{22.8}

        % the dimension of each rectangular element
        \def\xstep{1.0}
        \def\ystep{1.2}

        % Define clamping wall parameters
        \def\wallh{1.2}
        \def\wallt{1.0}

        % Define markers' diameter
        \def\noder{0.5}

        % Draw the side left clamping wall 
        \draw[pattern=north west lines] 
            (   \xmin, \ymin, -\wallh) -- 
            (   \xmin, \ymin,  \wallh) -- 
            ( -\wallt, \ymin,  \wallh) -- 
            ( -\wallt, \ymin, -\wallh) -- 
            cycle;

        % Draw the top left clamping wall
        \draw[pattern=north west lines] 
            (   \xmin, \ymin       ,  \wallh) -- 
            (   \xmin, \ymax+\ystep,  \wallh) -- 
            ( -\wallt, \ymax+\ystep,  \wallh) -- 
            ( -\wallt, \ymin       ,  \wallh) -- 
            cycle;

        % Draw the front left clamping wall
        \draw[fill=white]
            (   \xmin, \ymin       , -\wallh) -- 
            (   \xmin, \ymax+\ystep, -\wallh) --
            (   \xmin, \ymax+\ystep,  \wallh) --
            (   \xmin, \ymin       ,  \wallh) -- 
            cycle; 
    
        % Draw the left component
        \foreach \x in {\xmin,\xstep,...,\xmax}
            \foreach \y in {\ymin,\ystep,...,\ymax}
                \draw[fill=white] 
                      ( \x   , \y   , 0) -- 
                    ++( \xstep, 0    , 0) -- 
                    ++( 0    , \ystep, 0) -- 
                    ++(-\xstep, 0    , 0) -- 
                    cycle;

        \def\xmax{13.0}
        \def\offset{16.0}
        \foreach \x in {\xmin,\xstep,...,\xmax}
            \foreach \y in {\ymin,\ystep,...,\ymax}
                \draw[fill=white] 
                      ( \x + \offset + \gap  , \y, 0) -- 
                    ++( \xstep, 0    , 0) -- 
                    ++( 0    , \ystep, 0) -- 
                    ++(-\xstep, 0    , 0) -- 
                    cycle;

        \def\xmax{17.0}
        \def\offset{30.0}
        \foreach \x in {\xmin,\xstep,...,\xmax}
            \foreach \y in {\ymin,\ystep,...,\ymax}
                \draw[fill=white] 
                      ( \x + \offset + 2*\gap  , \y, 0) -- 
                    ++( \xstep, 0    , 0) -- 
                    ++( 0    , \ystep, 0) -- 
                    ++(-\xstep, 0    , 0) -- 
                    cycle;

        % Put the back right clamping wall in front of the grid
        \def\offset{48.0}
        \draw[fill=white]
            ( \offset+\wallt+2*\gap, \ymin       , -\wallh) -- 
            ( \offset+\wallt+2*\gap, \ymax+\ystep, -\wallh) --
            ( \offset+\wallt+2*\gap, \ymax+\ystep,  \wallh) --
            ( \offset+\wallt+2*\gap, \ymin       ,  \wallh) -- 
            cycle; 

        % Draw the back right clamping wall
        \draw[pattern=north east lines]
            ( \offset+\wallt+2*\gap, \ymin       , -\wallh) -- 
            ( \offset+\wallt+2*\gap, \ymax+\ystep, -\wallh) --
            ( \offset+\wallt+2*\gap, \ymax+\ystep,  \wallh) --
            ( \offset+\wallt+2*\gap, \ymin       ,  \wallh) -- 
            cycle; 

        % Put the side right clamping wall in front of the grid
        \draw[fill=white] 
            ( \offset+2*\gap       , \ymin, -\wallh) -- 
            ( \offset+2*\gap       , \ymin,  \wallh) -- 
            ( \offset+2*\gap+\wallt, \ymin,  \wallh) -- 
            ( \offset+2*\gap+\wallt, \ymin, -\wallh) -- 
            cycle;
        
        % Draw the side right clamping wall
        \draw[pattern=north west lines] 
            ( \offset+2*\gap       , \ymin, -\wallh) -- 
            ( \offset+2*\gap       , \ymin,  \wallh) -- 
            ( \offset+2*\gap+\wallt, \ymin,  \wallh) -- 
            ( \offset+2*\gap+\wallt, \ymin, -\wallh) -- 
            cycle;
        
        % Put the top right clamping wall in front of the grid
        \draw[fill=white] 
            ( \offset+2*\gap       , \ymin       ,  \wallh) -- 
            ( \offset+2*\gap       , \ymax+\ystep,  \wallh) -- 
            ( \offset+2*\gap+\wallt, \ymax+\ystep,  \wallh) -- 
            ( \offset+2*\gap+\wallt, \ymin       ,  \wallh) -- 
            cycle;

        % Draw the top right clamping wall
        \draw[pattern=north west lines] 
            ( \offset+2*\gap       , \ymin       ,  \wallh) -- 
            ( \offset+2*\gap       , \ymax+\ystep,  \wallh) -- 
            ( \offset+2*\gap+\wallt, \ymax+\ystep,  \wallh) -- 
            ( \offset+2*\gap+\wallt, \ymin       ,  \wallh) -- 
            cycle;

        % Draw a marker for the node B
        \draw[fill=red] 
            ( 8*\xstep, 9.5*\ystep ) ellipse (0.3 and 0.3);
        % Draw a line pointing at the marker for the node B
        \draw[thick]
            (  8*\xstep, 9.5*\ystep,        0 ) --
            ( -1*\xstep, 9.5*\ystep, 2*\wallh );
        % Add label B
        \node at
            ( -2*\xstep, 9.5*\ystep, 2*\wallh+0.6*\xstep ) {B};

        % Draw a marker for the node A
        \draw[fill=red] 
            ( 32*\xstep+2*\gap, 20*\ystep ) ellipse (0.3 and 0.3);
        % Draw a line pointing at the two markers for the node A
        \draw[thick]
            ( 32*\xstep+2*\gap         , 20*\ystep       , 0               ) --
            ( 32*\xstep+  \gap+7*\xstep, 20*\ystep+\ystep, \wallh+3*\xstep );
        % Add label A
        \node at (32*\xstep+8*\xstep+\gap,20*\ystep+\ystep, \wallh+4*\xstep) {A};

    \end{tikzpicture}
    \caption{Meshed Substructured Plate Model with Locations of Nodes A and B}
    \label{fig: substructured model}
\end{figure}
% -----------------------------------------------------------------------------
% Master thesis in the study program computational mechanics
%
% B.Sc. Rezha Adrian Tanuharja - 03751261
% M.Sc. Felix Schneider (supervisor)
%
% chapters/methodology/caseStudy/surrogateModel.tex
% Last edited 03 November 2023
% -----------------------------------------------------------------------------

\subsection{Sparse NI-RPCE Model}
\label{ssec: surrogate model}

This study uses RPCE models with products of probabilist Hermite polynomials as the basis functions, as in \eqref{prob Hermite products}.
The numerator is a PCE with an order of $5$ while the denominator is a PCE with an order of $6$.
Therefore, the basis functions in the numerator and denominator satisfy
\begin{equation}
    \Psi_{k}\left(\mathbf{\Xi}\right)
    =
    {He}_{r_{k}} \left(\xi_{1}\right)
    \cdot
    {He}_{s_{k}} \left(\xi_{2}\right)
    \phantom{x}
    \forall 
    \phantom{x}
    r_{k}, s_{k} \in \mathbb{N}, 
    \phantom{x}
    r_{k} + s_{k} \leq 5
\end{equation}
and
\begin{equation}
    \Psi_{l}\left(\mathbf{\Xi}\right)
    =
    {He}_{r_{l}} \left(\xi_{1}\right)
    \cdot
    {He}_{s_{l}} \left(\xi_{2}\right)
    \phantom{x}
    \forall 
    \phantom{x}
    r_{l}, s_{l} \in \mathbb{N}, 
    \phantom{x}
    r_{l} + s_{l} \leq 6,
\end{equation}
respectively.
The author obtains the coefficient in \eqref{RPCE_approx} by solving \eqref{SVD problem}.
There are $49$ basis functions in the numerator and denominator of the RPCE models.
The author reduces the number of basis functions by following the algorithms \ref{alg: sparse RPCE numerator} and \ref{alg: sparse RPCE denominator} with $\epsilon_{0}=1.0$.