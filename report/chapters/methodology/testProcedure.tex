% -----------------------------------------------------------------------------
% Master thesis in the study program computational mechanics
%
% B.Sc. Rezha Adrian Tanuharja - 03751261
% M.Sc. Felix Schneider (supervisor)
%
% chapters/methodology/testProcedure.tex
% Last edited 03 November 2023
% -----------------------------------------------------------------------------

\section{Evaluation Procedures}
\label{sec: evaluation procedure}

First, the author evaluates the performance of CUCB and HUCB in approximating the complete plate model.
The author generates an experimental design (ED) of size $N_{ed}=300$ through random sampling.
Subsequently, a direct MCS for FRFs using the complete plate model takes place.
The outputs serve as the benchmark for comparison.
Then, the author runs MCS for FRFs using the substructured plate model with the CUCB and HUCB methods.
Comparisons between the outputs with the benchmark and error calculations follow.
Because the size of the ED is not large, the author uses bootstrapping with a $1000$ number of resamplings to estimate the error statistics.
Figure \ref{flowchart: UCB procedure} illustrates this evaluation procedure.
% -----------------------------------------------------------------------------
% Master thesis in the study program computational mechanics
%
% B.Sc. Rezha Adrian Tanuharja - 03751261
% M.Sc. Felix Schneider (supervisor)
%
% images/procedureUCB.tex
% Last edited 03 November 2023
% -----------------------------------------------------------------------------

\begin{figure}[H]
    \centering
    \scalebox{0.75}{
        \begin{tikzpicture}[
            startstop/.style={%
                thick, rectangle, rounded corners,%
                minimum width=\charthgap, minimum height=\chartvgap,%
                text centered, draw=black, fill=white, text width=\charttwid%
            },
            io/.style={%
                thick, trapezium, trapezium left angle=75, trapezium right angle=105,%
                minimum width=\charthgap, minimum height=\chartvgap,%
                text centered, draw=black, fill=white, text width=\chartvgap%
            },
            process/.style={%
                thick, rectangle,%
                minimum width=\charthgap, minimum height=\chartvgap,%
                text centered, draw=black, fill=white, text width=\charttwid%
            },
            blankprocess/.style={%
                thick, rectangle,%
                minimum width=\charthgap, minimum height=\chartvgap,%
                text centered, draw=none, fill=none, text width=\charttwid%
            },
            decision/.style={%
                thick, diamond,%
                minimum width=\chartvgap, minimum height=\chartvgap,%
                text centered, draw=black, fill=white, text width=\chartvgap%
            },
            arrow/.style={thick,->,>=stealth}
        ]

        % Define nodes
        \node (start) [startstop] {%
            Start
        };

        \node (input1) [
                io, below of=start,
                % xshift=-0.7*\charthgap,
                yshift=-\chartvgap
            ] {$\phantom{a}$};

        \node (ED_1) [
                blankprocess, below of=start,
                % xshift=-0.7*\charthgap,
                yshift=-\chartvgap
            ] {
            Experimental Design
            };

        \node (fullEval) [
                process, below of=ED_1,
                xshift=-0.7*\charthgap,
                yshift=-\chartvgap
            ] {
                MC Simulations\\
                Complete Plate Model
            };

        \node (subEval) [
                process, below of=ED_1,
                xshift=0.7*\charthgap,
                yshift=-\chartvgap
            ] {
                MC Simulations\\
                CUCB / HUCB
            };

        \node (input5) [io, below of=subEval, yshift=-\chartvgap] {
            $\phantom{a}$
        };
        \node (ED_5) [blankprocess, below of=subEval, yshift=-\chartvgap] {
            Output Estimates\\
            CUCB / HUCB
        };

        \node (input6) [io, below of=fullEval, yshift=-\chartvgap] {
            $\phantom{a}$
        };
        \node (ED_6) [blankprocess, below of=fullEval, yshift=-\chartvgap] {
            Output Benchmarks
        };

        \path 
            (ED_5) -- (ED_6) coordinate[midway] (ED_56);

        \node (compare) [process, below of=ED_56, yshift=-\chartvgap] {
            Bootstrapping
        };
        \node (iterate) [process, below of=compare, yshift=-\chartvgap] {
            Comparison
        };
        \node (stop) [startstop, below of=iterate, yshift=-1.2*\chartvgap] {Stop};

        \draw [arrow] (start) -- (ED_1);
        \draw [arrow] (input1) -| (fullEval);
        \draw [arrow] (input1) -| (subEval);
        \draw [arrow] (fullEval) -- (ED_6);
        \draw [arrow] (subEval) -- (ED_5);
        \draw [arrow] (ED_6) |- (compare);
        \draw [arrow] (ED_5) |- (compare);
        \draw [arrow] (compare) -- (iterate);
        \draw [arrow] (iterate) -- (stop);

        % \path (start) -- (ED_1) coordinate[midway] (midpoint_1);
        % \draw [arrow] (midpoint_1) -| (ED_2);
        % \draw [arrow] (ED_2) -- (subEval);
        % \draw [arrow] (subEval) -- (ED_3);
        % \draw [arrow] (ED_3) -- (resample);
        % \draw [arrow] (resample) -- (ED_4);
        % \draw [arrow] (ED_4) -- (train);
        % \draw [arrow] (train) -- (evalRPCE);
        % \draw [arrow] (evalRPCE) -- (ED_5);
        % \draw [arrow] (ED_5) |- (compare);

        % \draw [arrow] (compare) -- (iterate);
        % \draw [arrow] (iterate) -- node[right] {n} (stop) ;

        % \draw [thick] 
        %     (iterate) -- 
        %     ++(1.2*\chartvgap,0) node[midway, above] {y} coordinate (endpoint_1);
        % \draw [thick] 
        %     (endpoint_1) -- node[above] {$i=i+1$}
        %     ++(1.0*\charthgap,0) coordinate (endpoint_2);

        % \draw [arrow] 
        %     (endpoint_2) |- (resample);

        % \path 
        %     (ED_1) -- (fullEval) coordinate[midway] (point_1);
        % \path 
        %     (ED_2) -- (subEval) coordinate[midway] (point_2);
        % \path 
        %     (point_1) -- (point_2) coordinate[midway] (point_3);
        % \draw [thick] 
        %     (point_1) -- (point_3);
        % \draw [arrow] 
        %     (point_3) |- (evalRPCE);

        \end{tikzpicture}
    }
    \caption{The UCBs Evaluation Procedure Flowchart}
    \label{flowchart: UCB procedure}
\end{figure}

Subsequently, the author evaluates the performance of HUCB + NI-RPCE models in approximating the complete plate model.
The author generates two EDs of size $N_{ed}=300$ through random sampling.
The first ED is for direct MCS using the complete plate model.
Similar to the procedure above, the outputs serve as the benchmark for comparison.
Then, the author runs MCS for FRFs using the substructured plate model with the HUCB methods, using the second ED.
The outputs serve as the pool of training data for the RPCE models.
The training process of the RPCE models follows, using a random subset of these training data.
The author then feeds the first ED to the trained model.
Afterward, the comparisons between the outputs with the benchmark and error calculations follow.
To take into account the effect of the training data, the author varies the size of the random subset and repeats these steps $N=40$ times for each subset's size.
Figure \ref{flowchart: framework procedure} illustrates the evaluation procedure.
% -----------------------------------------------------------------------------
% Master thesis in the study program computational mechanics
%
% B.Sc. Rezha Adrian Tanuharja - 03751261
% M.Sc. Felix Schneider (supervisor)
%
% images/procedureFramework.tex
% Last edited 03 November 2023
% -----------------------------------------------------------------------------

\begin{figure}[H]
    \centering
    \scalebox{0.75}{
        \begin{tikzpicture}[
            startstop/.style={%
                thick, rectangle, rounded corners,%
                minimum width=\charthgap, minimum height=\chartvgap,%
                text centered, draw=black, fill=white, text width=\charttwid%
            },
            io/.style={%
                thick, trapezium, trapezium left angle=75, trapezium right angle=105,%
                minimum width=\charthgap, minimum height=\chartvgap,%
                text centered, draw=black, fill=white, text width=\chartvgap%
            },
            process/.style={%
                thick, rectangle,%
                minimum width=\charthgap, minimum height=\chartvgap,%
                text centered, draw=black, fill=white, text width=\charttwid%
            },
            blankprocess/.style={%
                thick, rectangle,%
                minimum width=\charthgap, minimum height=\chartvgap,%
                text centered, draw=none, fill=none, text width=\charttwid%
            },
            decision/.style={%
                thick, diamond,%
                minimum width=\chartvgap, minimum height=\chartvgap,%
                text centered, draw=black, fill=white, text width=\chartvgap%
            },
            arrow/.style={thick,->,>=stealth}
        ]

        % Define nodes
        \node (start) [startstop] {%
            Start
        };
        \node (input1) [io, below of=start, yshift=-\chartvgap] {$\phantom{a}$};
        \node (ED_1) [blankprocess, below of=start, yshift=-\chartvgap] {
            Experimental Design 1
            % $\left\{\mathbf{X}_{1}, ..., \mathbf{X}_{n}\right\}$
        };
        \node (fullEval) [process, below of=ED_1, yshift=-\chartvgap] {
            MC Simulations\\
            Complete Plate Model
        };
        \node (input2) [io, right of=ED_1, xshift=\charthgap] {$\phantom{a}$};
        \node (ED_2) [blankprocess, right of=ED_1, xshift=\charthgap] {
            Experimental Design 2
            % $\left\{\widetilde{\mathbf{X}}_{1}, ..., 
            % \widetilde{\mathbf{X}}_{n}\right\}$
        };
        \node (subEval) [process, below of=ED_2, yshift=-\chartvgap] {
            MC Simulations\\
            HUCB 
        };
        \node (input3) [io, below of=subEval, yshift=-\chartvgap] {
            $\phantom{a}$
        };
        \node (ED_3) [blankprocess, below of=subEval, yshift=-\chartvgap] {
            Training Data
            % $\widetilde{\mathcal{M}}\left(
            %     \widetilde{\mathbf{X}}_{1}
            % \right), ...,
            % \widetilde{\mathcal{M}}\left(
            %     \widetilde{\mathbf{X}}_{n}
            % \right)$
        };
        \node (resample) [process, below of=ED_3, yshift=-\chartvgap] {
            Random Selection of Training Data
        };
        \node (input4) [io, below of=resample, yshift=-\chartvgap] {
            $\phantom{a}$
        };
        \node (ED_4) [blankprocess, below of=resample, yshift=-\chartvgap] {
            Subset of Training Data
        };
        \node (train) [process, below of=ED_4, yshift=-\chartvgap] {
            Train NI-RPCE Models\\
            Regular / Sparse Models
        };
        \node (evalRPCE) [process, below of=train, yshift=-\chartvgap] {
            Evaluate NI-RPCE Model Using ED 1
        };
        \node (input5) [io, below of=evalRPCE, yshift=-\chartvgap] {
            $\phantom{a}$
        };
        \node (ED_5) [blankprocess, below of=evalRPCE, yshift=-\chartvgap] {
            Output Estimates\\
            Proposed Framework
        };
        \node (input6) [io, left of=ED_5, xshift=-\charthgap] {
            $\phantom{a}$
        };
        \node (ED_6) [blankprocess, left of=ED_5, xshift=-\charthgap] {
            Output Benchmarks
        };

        \path 
            (ED_5) -- (ED_6) coordinate[midway] (ED_56);

        \node (compare) [process, below of=ED_56, yshift=-\chartvgap] {
            Comparison
        };
        \node (iterate) [decision, below of=compare, yshift=-1.2*\chartvgap] {
            $i < N$
        };
        \node (stop) [startstop, below of=iterate, yshift=-1.2*\chartvgap] {Stop};

        \draw [arrow] (start) -- (ED_1);
        \draw [arrow] (ED_1) -- (fullEval);
        \draw [arrow] (fullEval) -- (ED_6);
        \draw [arrow] (ED_6) |- (compare);

        \path (start) -- (ED_1) coordinate[midway] (midpoint_1);
        \draw [arrow] (midpoint_1) -| (ED_2);
        \draw [arrow] (ED_2) -- (subEval);
        \draw [arrow] (subEval) -- (ED_3);
        \draw [arrow] (ED_3) -- (resample);
        \draw [arrow] (resample) -- (ED_4);
        \draw [arrow] (ED_4) -- (train);
        \draw [arrow] (train) -- (evalRPCE);
        \draw [arrow] (evalRPCE) -- (ED_5);
        \draw [arrow] (ED_5) |- (compare);

        \draw [arrow] (compare) -- (iterate);
        \draw [arrow] (iterate) -- node[right] {n} (stop) ;

        \draw [thick] 
            (iterate) -- 
            ++(1.2*\chartvgap,0) node[midway, above] {y} coordinate (endpoint_1);
        \draw [thick] 
            (endpoint_1) -- node[above] {$i=i+1$}
            ++(1.0*\charthgap,0) coordinate (endpoint_2);

        \draw [arrow] 
            (endpoint_2) |- (resample);

        \path 
            (ED_1) -- (fullEval) coordinate[midway] (point_1);
        \path 
            (ED_2) -- (subEval) coordinate[midway] (point_2);
        \path 
            (point_1) -- (point_2) coordinate[midway] (point_3);
        \draw [thick] 
            (point_1) -- (point_3);
        \draw [arrow] 
            (point_3) |- (evalRPCE);

        \end{tikzpicture}
    }
    \caption{The Proposed Framework Evaluation Procedure Flowchart}
    \label{flowchart: framework procedure}
\end{figure}