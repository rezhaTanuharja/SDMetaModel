% -----------------------------------------------------------------------------
% Master thesis in the study program computational mechanics
%
% B.Sc. Rezha Adrian Tanuharja - 03751261
% M.Sc. Felix Schneider (supervisor)
%
% chapters/literature/stochastic.tex
% Last edited 03 November 2023
% -----------------------------------------------------------------------------

\section{Uncertainty Propagation}
\label{sec: stochastic}

The previous section considers deterministic dynamic structures, i.e., it assumes that all structural parameters are precisely known.
This assumption is not necessarily valid.
{Alternatively, one can use a dynamic model with parametric uncertainties instead:}%
\begin{equation}
    \mathbf{D} \left(\omega, \mathbf{\Xi}\right)
    \mathbf{U} \left(\omega, \mathbf{\Xi}\right)
    =
    \mathbf{F} \left(\omega, \mathbf{\Xi}\right),
    \label{stochastic_eq_of_motion}
\end{equation}
where the dynamic stiffness, force vector, and displacement vector are functions of a set of zero-mean random variables $\mathbf{\Xi}=\left\{\xi_{1}, \xi_{2},...,\xi_{m}\right\}$.
In this context, the outputs of the model are not deterministic.
Therefore, the aim is to estimate the probabilistic characteristics of the model's outputs.

% -----------------------------------------------------------------------------
% Master thesis in the study program computational mechanics
%
% B.Sc. Rezha Adrian Tanuharja - 03751261
% M.Sc. Felix Schneider (supervisor)
%
% chapters/literature/stochastic/monteCarlo.tex
% Last edited 03 November 2023
% -----------------------------------------------------------------------------

\subsection{Monte Carlo Method}
\label{ssec: MC method}

The MC method is a classic uncertainty quantification approach.
It involves generating random samples of $\mathbf{\Xi}$, evaluating the model for each realization, and aggregating the results to estimate the probabilistic characteristics of the model outputs.
Algorithm \ref{alg: MC method} illustrates an MC simulation (MCS) for an uncertainty quantification of FRFs.
% -----------------------------------------------------------------------------
% Master thesis in the study program computational mechanics
%
% B.Sc. Rezha Adrian Tanuharja - 03751261
% M.Sc. Felix Schneider (supervisor)
%
% chapters/literature/stochastic/monteCarlo.tex
% Last edited 03 November 2023
% -----------------------------------------------------------------------------

\subsection{Monte Carlo Method}
\label{ssec: MC method}

The MC method is a classic uncertainty quantification approach.
It involves generating random samples of $\mathbf{\Xi}$, evaluating the model for each realization, and aggregating the results to estimate the probabilistic characteristics of the model outputs.
Algorithm \ref{alg: MC method} illustrates an MC simulation (MCS) for an uncertainty quantification of FRFs.
% -----------------------------------------------------------------------------
% Master thesis in the study program computational mechanics
%
% B.Sc. Rezha Adrian Tanuharja - 03751261
% M.Sc. Felix Schneider (supervisor)
%
% chapters/literature/stochastic/monteCarlo.tex
% Last edited 03 November 2023
% -----------------------------------------------------------------------------

\subsection{Monte Carlo Method}
\label{ssec: MC method}

The MC method is a classic uncertainty quantification approach.
It involves generating random samples of $\mathbf{\Xi}$, evaluating the model for each realization, and aggregating the results to estimate the probabilistic characteristics of the model outputs.
Algorithm \ref{alg: MC method} illustrates an MC simulation (MCS) for an uncertainty quantification of FRFs.
\input{algorithms/monteCarlo}
% -----------------------------------------------------------------------------
% Master thesis in the study program computational mechanics
%
% B.Sc. Rezha Adrian Tanuharja - 03751261
% M.Sc. Felix Schneider (supervisor)
%
% algorithm/PCE.tex
% Last edited 03 November 2023
% -----------------------------------------------------------------------------

\begin{center}
\begin{algorithm}[H]
    \label{alg: PCE}
    \ForEach{$\omega$ in $\left\{\omega_{1}, ..., \omega_{m}\right\}$}{
        \vspace{1.0em}%
        $\mathbf{Y} \longleftarrow [\phantom{x}]$ \\ 
        $\mathbf{A} \longleftarrow [\phantom{x}]$ \\ 
        \vspace{1.0em}%
        \tcp{Generate random samples}
        \For{$i = 1, ..., N_{sample}$}{
            $\mathbf{\Xi}_{i} \longleftarrow$
            \sample{$\mathbf{\Xi}$} \\
            \vspace{1.0em}%
            \tcc{Compute dynamic stiffness matrix $\mathbf{D}\left(\omega, \mathbf{\Xi}_{i}\right)$}
            $\dots$ \\
            \vspace{1.0em}%
            $
                \mathbf{Y}\left[i\right] \longleftarrow
                -\omega^{2}\left(\mathbf{D}\left(\omega, \mathbf{\Xi}_{i}\right)\right)^{-1}
            $ \\
            $
                \mathbf{A}\left[i\right] \longleftarrow
                \left[\Psi_{k}\left(\mathbf{\Xi}_{i}\right),\;\; k \in S\right]
            $
        }
        \vspace{1.0em}%
        $\left\{\mathbf{H}_{k}, k\in S\right\} \longleftarrow$ 
        \lsRegression{$\mathbf{Y}, \mathbf{A}$} \\
        \vspace{1.0em}%
        $\hat{\mathbf{H}} \longleftarrow [\phantom{x}]$ \\ 
        \vspace{1.0em}%
        \tcp{Perform MC simulation using PCE}
        \For{$i = 1, ..., N_{mcs}$}{
            $\mathbf{\Xi}_{i} \longleftarrow$
            \sample{$\mathbf{\Xi}$} \\
            \vspace{1.0em}%
            \tcp{Approximate FRF}
            $
                \hat{\mathbf{H}}\left[i\right] \longleftarrow
            $
                \sum{
                    $
                    \Psi_{k} \left(\mathbf{\Xi}_{i}\right)
                    \cdot 
                    \mathbf{H}_{k}
                    $
                }
            % $
        }
        \vspace{1.0em}%
        \tcc{Estimate characteristics of $\mathbf{H}\left(\omega, \mathbf{\Xi}\right)$ from $\hat{\mathbf{H}}$}
        $\dots$
    }
  \caption{Monte Carlo Simulation of The FRFs with PCE}
\end{algorithm}
\end{center}
% -----------------------------------------------------------------------------
% Master thesis in the study program computational mechanics
%
% B.Sc. Rezha Adrian Tanuharja - 03751261
% M.Sc. Felix Schneider (supervisor)
%
% chapters/literature/stochastic/sparsePCE.tex
% Last edited 03 November 2023
% -----------------------------------------------------------------------------

\subsection{Sparse PCE Model}
\label{ssec: sparse PCE}

The computation of FRFs is likely the most expensive step in the method above.
Therefore, the number of samples dictates the overall computational cost.
As the number of basis functions in the approximation increases, so does the required number of samples.
There are several procedures to shrink the number of basis functions.
The following is an example of such procedures and is a simplified version of the one in \cite{blatman2010adaptive}.

The procedure starts with an empty approximation.
It enriches the model with one basis function at a time and checks its accuracy.
If accuracy improves significantly, it keeps the basis function in the model.
Otherwise, it removes the basis functions from the model.
These steps aim to include only the basis functions that significantly improve the model's approximation.
Algorithm \ref{alg: sparse PCE addition} illustrates these steps.

% -----------------------------------------------------------------------------
% Master thesis in the study program computational mechanics
%
% B.Sc. Rezha Adrian Tanuharja - 03751261
% M.Sc. Felix Schneider (supervisor)
%
% algorithm/sparsePCEAddition.tex
% Last edited 03 November 2023
% -----------------------------------------------------------------------------

\begin{center}
\begin{algorithm}[H]
    \label{alg: sparse PCE addition}
    \vspace{1.0em}%
    \tcp{Initialize with an empty set of indices}
    $\mathbf{S} \longleftarrow \left\{\phantom{x}\right\}$ \\
    \vspace{1.0em}%
    \tcp{Compute initial error when all coefficients are zero}
    $\epsilon \longleftarrow$
    \err{$\mathbf{Y}_{test}, \mathbf{A}_{test}, \mathbf{0}$} \\
    \vspace{1.0em}%
    \For{$i = 1, ..., N_{p}$}{
        \vspace{1.0em}%
        $S \longleftarrow S\cup\left\{i\right\}$ \\
        \vspace{1.0em}%
        \tcp{Only compute coefficients with index in $S$}
        $\mathbf{x} \:\phantom{\left[S\right]} \longleftarrow$
        \zeros{$N_{p}$} \\
        $\mathbf{x}\left[S\right] \longleftarrow$
        \lsRegression{$\mathbf{Y}_{train}, \mathbf{A}_{train}\left[\;:,\;S\;\right]$} \\
        \vspace{1.0em}%
        $\epsilon^{+} \longleftarrow$
        \err{$\mathbf{Y}_{test}, \mathbf{A}_{test}, \mathbf{x}$} \\
        \vspace{1.0em}%
        \tcp{Keep $i$ in $S$ if error reduction is significant}
        \uIf{$\epsilon -\epsilon^{+} > \epsilon_{0}$}{
            $\epsilon \longleftarrow \epsilon^{+}$
        }
        \Else{
            $S\setminus \left\{i\right\}$
        }
        \vspace{1.0em}%
    }
    \vspace{1.0em}%
  \caption{Basis Addition Steps for Sparse PCE Model}
\end{algorithm}
\end{center}

Subsequently, the procedure removes one basis function at a time from the approximation and checks the accuracy.
If accuracy declines significantly, it reinstates the basis in the approximation.
These steps aim to remove the basis functions that do not significantly improve the model's approximation.
Algorithm \ref{alg: sparse PCE removal} illustrates these steps.

% -----------------------------------------------------------------------------
% Master thesis in the study program computational mechanics
%
% B.Sc. Rezha Adrian Tanuharja - 03751261
% M.Sc. Felix Schneider (supervisor)
%
% algorithm/sparsePCERemoval.tex
% Last edited 03 November 2023
% -----------------------------------------------------------------------------

\begin{center}
\begin{algorithm}[H]
    \label{alg: sparse PCE removal}
    \vspace{1.0em}%
    \ForEach{$i \in S$}{
        \vspace{1.0em}%
        $S\setminus \left\{i\right\}$ \\
        \vspace{1.0em}%
        \tcp{Only compute coefficients with index in $S$}
        $\mathbf{x} \:\phantom{\left[S\right]} \longleftarrow$
        \zeros{$N_{p}$} \\
        $\mathbf{x}\left[S\right] \longleftarrow$
        \lsRegression{$\mathbf{Y}_{train}, \mathbf{A}_{train}\left[\;:,\;S\;\right]$} \\
        \vspace{1.0em}%
        $\epsilon^{-} \longleftarrow$
        \err{$\mathbf{Y}_{test}, \mathbf{A}_{test}, \mathbf{x}$} \\
        \vspace{1.0em}%
        \tcp{Reinstate $i$ in $S$ if error increase is significant}
        \uIf{$\epsilon^{-} -\epsilon > \epsilon_{0}$}{
            $S \longleftarrow S\cup\left\{i\right\}$
        }
        \Else{
            $\epsilon \longleftarrow \epsilon^{-}$
        }
        \vspace{1.0em}%
    }
    \vspace{1.0em}%
  \caption{Basis Removal Steps for Sparse PCE Model}
\end{algorithm}
\end{center}

After the selective inclusion and exclusion steps above, only some of the basis functions are present in the PCE model.
Hence the term sparse PCE model.
% -----------------------------------------------------------------------------
% Master thesis in the study program computational mechanics
%
% B.Sc. Rezha Adrian Tanuharja - 03751261
% M.Sc. Felix Schneider (supervisor)
%
% chapters/literature/stochastic/RPCE.tex
% Last edited 03 November 2023
% -----------------------------------------------------------------------------

\subsection{Rational PCE Model}
\label{ssec: RPCE}

PCE requires a large number of basis functions to accurately represent the highly nonlinear behavior of FRFs near the structure's eigenfrequencies.
As the subsection above mentions, the large number of basis functions leads to a large number of required samples, thus increasing the overall computational cost.
{An alternative is approximating FRFs with ratios of PCEs:}%
\begin{equation}
    H_{pq} \left( \omega, \mathbf{\Xi} \right)
    \approx
    \frac{
        \sum_{k\in S_{u}}{
            u_{pqk}\left(\omega\right)
            \cdot
            \Psi_{k} \left( \mathbf{\Xi} \right)
        }
    }{
        \sum_{l\in S_{v}}{
            v_{pql}\left(\omega\right)
            \cdot
            \Psi_{l} \left( \mathbf{\Xi} \right)
        }
    }
    \quad
    S_{u} = \left\{1, ..., N_{u}\right\},
    \;
    S_{v} = \left\{1, ..., N_{v}\right\}
    .
    \label{RPCE_approx}
\end{equation}
The mean square error is typically the prime candidate to be an objective function in a linear regression problem.
However, the rational form of RPCE makes finding its minima less trivial.
{In \cite{schneider2020polynomial}, the authors define an alternative objective function for the minimization problem:}%
\begin{equation}
    \left\{
        \mathbf{u}_{pq},
        \mathbf{v}_{pq}
    \right\}
    =
    \underset{
        \left\{
            \hat{\mathbf{u}},
            \hat{\mathbf{v}}
        \right\}
        \in 
        \mathbb{C}^{N_{u}+N_{v}}
    }{\arg\min}
    \;
    \frac{1}{N}
    \sum_{i=0}^{N}
    \left\|
        H_{pq} \left( \mathbf{\Xi}_{i} \right)
        \cdot
        \sum_{l\in S_{v}}{
            \hat{v}_{l} \Psi_{l} \left( \mathbf{\Xi}_{i} \right)
        }
        -
        \sum_{k\in S_{u}}{
            \hat{u}_{k} \Psi_{k} \left( \mathbf{\Xi}_{i} \right)
        }
    \right\|_{2}^{2},
\end{equation}
where the dependency on frequency is intentionally missing to keep it concise.
{Subsequently, they show that the solution satisfies}%
\begin{equation}
    \begin{bmatrix}
        \mathbf{\Psi}_{U}^{T}
        \mathbf{\Psi}_{U}
        &
        -\mathbf{\Psi}_{U}^{T}
        \mathbf{M}
        \mathbf{\Psi}_{V}
        \\
        -\mathbf{\Psi}_{V}^{T}
        \mathbf{M}^{H}
        \mathbf{\Psi}_{U}
        &
        \mathbf{\Psi}_{V}^{T}
        \mathbf{M}^{H}
        \mathbf{M}
        \mathbf{\Psi}_{V}
    \end{bmatrix}
    \begin{pmatrix}
        \mathbf{u}_{pq} \\
        \mathbf{v}_{pq}
    \end{pmatrix}
    =
    \begin{pmatrix}
        \mathbf{0} \\
        \mathbf{0}
    \end{pmatrix},
    \label{SVD problem}
\end{equation}
{where}%
\begin{equation}
    \left[\mathbf{\Psi}_{U}\right]_{ij}
    =
    \Psi_{j} \left(\mathbf{\Xi}_{i}\right)
    \quad
    \forall
    \quad
    (i,j)\in\left\{1, ..., N\right\} \otimes
    \left\{1, ..., N_{u}\right\},
\end{equation}
\vspace{-3.0em}
\begin{equation}
    \left[\mathbf{\Psi}_{V}\right]_{ij}
    =
    \Psi_{j} \left(\mathbf{\Xi}_{i}\right)
    \quad
    \forall
    \quad
    (i,j)\in\left\{1, ..., N\right\} \otimes
    \left\{1, ..., N_{v}\right\},
\end{equation}
{and}%
\begin{equation}
    \mathbf{M}
    =
    \text{diag} \left(
        H_{pq} \left( \mathbf{\Xi}_{1} \right),
        H_{pq} \left( \mathbf{\Xi}_{2} \right),
        ..., 
        H_{pq} \left( \mathbf{\Xi}_{N} \right)
    \right).
\end{equation}