% -----------------------------------------------------------------------------
% Master thesis in the study program computational mechanics
%
% B.Sc. Rezha Adrian Tanuharja - 03751261
% M.Sc. Felix Schneider (supervisor)
%
% images/caseStudy.tex
% Last edited 03 November 2023
% -----------------------------------------------------------------------------

\begin{figure}[H]
    \centering

    \begin{tikzpicture}[
        scale=0.35, 
        %
        % Isometric-esque appearance
        x={( 0.6cm,-0.2cm)}, 
        y={( 0.5cm, 0.3cm)}, 
        z={( 0.0cm, 1.0cm)}, 
        %
        draw opacity=0.7
    ]

        % Define plate parameters
        % maxes are lengths - one step because of plotting
        \def\xmin{0} \def\xmax{47.0}
        \def\ymin{0} \def\ymax{22.8}
        % the dimension of each rectangular element
        \def\xstep{1.0}
        \def\ystep{1.2}

        % Define clamping wall parameters
        \def\wallh{1.2}
        \def\wallt{1.0}

        % Draw the side left clamping wall 
        \draw[pattern=north west lines] 
            (   \xmin, \ymin, -\wallh) -- 
            (   \xmin, \ymin,  \wallh) -- 
            ( -\wallt, \ymin,  \wallh) -- 
            ( -\wallt, \ymin, -\wallh) -- 
            cycle;

        % Draw the top left clamping wall
        \draw[pattern=north west lines] 
            (   \xmin, \ymin       ,  \wallh) -- 
            (   \xmin, \ymax+\ystep,  \wallh) -- 
            ( -\wallt, \ymax+\ystep,  \wallh) -- 
            ( -\wallt, \ymin       ,  \wallh) -- 
            cycle;

        % Draw the front left clamping wall
        \draw[fill=white]
            (   \xmin, \ymin       , -\wallh) -- 
            (   \xmin, \ymax+\ystep, -\wallh) --
            (   \xmin, \ymax+\ystep,  \wallh) --
            (   \xmin, \ymin       ,  \wallh) -- 
            cycle; 
    
        % Draw the plate
        \draw[fill=white] 
            ( \xmin       , \ymin       , 0) -- 
            ( \xmax+\xstep, \ymin       , 0) -- 
            ( \xmax+\xstep, \ymax+\ystep, 0) -- 
            ( \xmin       , \ymax+\ystep, 0) -- 
            cycle;

        % Draw annotations
        \draw[thin]
            ( \xmin, \ymax+  \ystep, 0 ) --
            ( \xmin, \ymax+4*\ystep, 0);

        \draw[thin]
            ( \xmax+\xstep, \ymax+  \ystep, 0 ) --
            ( \xmax+\xstep, \ymax+4*\ystep, 0);

        \draw[latex-latex]
            ( \xmin       , \ymax+3*\ystep, 0) --
            node[sloped, anchor=center, above] {$8.0\text{ m}$}
            ( \xmax+\xstep, \ymax+3*\ystep, 0);

        \draw[latex-latex]
            ( \xmax-5*\xstep, \ymin, 0 ) --
            node[sloped, anchor=center, above] {$4.0\text{ m}$}
            ( \xmax-5*\xstep, \ymax+\ystep, 0 );

        % Put the back right clamping wall in front of the plate
        \draw[fill=white]
            (   \xmax+\wallt+\xstep, \ymin       , -\wallh) -- 
            (   \xmax+\wallt+\xstep, \ymax+\ystep, -\wallh) --
            (   \xmax+\wallt+\xstep, \ymax+\ystep,  \wallh) --
            (   \xmax+\wallt+\xstep, \ymin       ,  \wallh) -- 
            cycle; 

        % Draw the back right clamping wall
        \draw[pattern=north east lines]
            (   \xmax+\wallt+\xstep, \ymin       , -\wallh) -- 
            (   \xmax+\wallt+\xstep, \ymax+\ystep, -\wallh) --
            (   \xmax+\wallt+\xstep, \ymax+\ystep,  \wallh) --
            (   \xmax+\wallt+\xstep, \ymin       ,  \wallh) -- 
            cycle; 

        % Put the side right clamping wall in front of the plate
        \draw[fill=white] 
            ( \xmax+\xstep       , \ymin, -\wallh) -- 
            ( \xmax+\xstep       , \ymin,  \wallh) -- 
            ( \xmax+\xstep+\wallt, \ymin,  \wallh) -- 
            ( \xmax+\xstep+\wallt, \ymin, -\wallh) -- 
            cycle;
        
        % Draw the side right clamping wall
        \draw[pattern=north west lines] 
            ( \xmax+\xstep       , \ymin, -\wallh) -- 
            ( \xmax+\xstep       , \ymin,  \wallh) -- 
            ( \xmax+\xstep+\wallt, \ymin,  \wallh) -- 
            ( \xmax+\xstep+\wallt, \ymin, -\wallh) -- 
            cycle;
        
        % Put the top right clamping wall in front of the plate
        \draw[fill=white] 
            ( \xmax+\xstep       , \ymin       ,  \wallh) -- 
            ( \xmax+\xstep       , \ymax+\ystep,  \wallh) -- 
            ( \xmax+\xstep+\wallt, \ymax+\ystep,  \wallh) -- 
            ( \xmax+\xstep+\wallt, \ymin       ,  \wallh) -- 
            cycle;

        % Draw the top right clamping wall
        \draw[pattern=north west lines] 
            ( \xmax+\xstep       , \ymin       ,  \wallh) -- 
            ( \xmax+\xstep       , \ymax+\ystep,  \wallh) -- 
            ( \xmax+\xstep+\wallt, \ymax+\ystep,  \wallh) -- 
            ( \xmax+\xstep+\wallt, \ymin       ,  \wallh) -- 
            cycle;

        % Draw coordinate axis
        \draw[thick, ->]
            ( \xmin, \ymin-6*\ystep, 0) --
            ( \xmin, \ymin-6*\ystep, -1.5*\xstep)
            node [below] {$u$};

        \draw[thick, ->]
            ( \xmin         , \ymin-6*\ystep, 0) --
            ( \xmin+2*\xstep, \ymin-6*\ystep, 0)
            node [right, xshift=-1.5*\xstep, yshift=-1.5*\ystep] {$x$};

        \draw[thick, ->]
            ( \xmin, \ymin-6*\ystep, 0) --
            ( \xmin, \ymin-4*\ystep, 0)
            node [above, xshift=5.5*\xstep, yshift=-2.5*\ystep] {$y$};

    \end{tikzpicture}
    \caption{Case Study: A Plate Clamped at Both Ends}
    \label{fig: case study}
\end{figure}