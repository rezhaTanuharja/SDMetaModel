% -----------------------------------------------------------------------------
% Master thesis in the study program computational mechanics
%
% B.Sc. Rezha Adrian Tanuharja - 03751261
% M.Sc. Felix Schneider (supervisor)
%
% chapters/development/sparseRPCE/inclusionCriteria.tex
% Last edited 03 November 2023
% -----------------------------------------------------------------------------

\subsection{Basis Inclusion Criteria}
\label{ssec: inclusion criteria}

The inclusion or exclusion of the basis functions is based on their contribution to the approximation's quality.
A metric to quantify the approximation's quality is necessary to do this.
{This study uses the coefficient of determination:}%
\begin{equation}
    R^{2}
    =
    1 -
    \left(1-\frac{1}{n}\right)
    \cdot
    \frac{
        \left\|\mathbf{x}-\hat{\mathbf{x}}\right\|_{2}^{2}
    }{
        \left\|\mathbf{x}-\overline{\mathbf{x}}\right\|_{2}^{2}
    },
\end{equation}
where $\mathbf{x}$ is the actual model outputs, $\overline{\mathbf{x}}$ is the average of the model outputs, and $\hat{\mathbf{x}}$ is the approximation.
This study defines a model to be better than another if it has a higher coefficient of determination.
A perfect model, i.e., a model that can produce $\hat{\mathbf{x}}=\mathbf{x}$ has a coefficient of determination of unity.