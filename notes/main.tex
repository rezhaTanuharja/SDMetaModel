\documentclass[a4paper,oneside,10pt]{article}
\setlength{\parindent}{0pt}

\usepackage{amsmath, amsthm, amssymb}
\usepackage{appendix}
\usepackage{array}
\usepackage{blindtext}
\usepackage{csquotes}
\usepackage{enumitem}
\usepackage{fancyhdr}
\usepackage{float}
\usepackage{graphicx}
\usepackage{here} 
\usepackage{listings}
\usepackage{lmodern} 
\usepackage{makeidx}
\usepackage{mathptmx}
\usepackage{mathrsfs}
\usepackage{microtype}
\usepackage{multicol}
\usepackage{nopageno}
\usepackage{subfig}
\usepackage{subfiles} 
\usepackage{txfonts}
\usepackage{varioref}
\usepackage{xcolor}
\usepackage[a4paper,vmargin={25mm,30mm},hmargin={30mm,20mm},footnotesep=1cm]{geometry}
\usepackage[english]{babel}
\usepackage[onehalfspacing]{setspace} 
\usepackage[section]{placeins}
\usepackage[style=numeric,sortcites=true,block=nbpar,backend=bibtex8]{biblatex}
\usepackage[T1]{fontenc}
\graphicspath{{images/}}

\lstdefinestyle{cpp}{
    belowcaptionskip=1\baselineskip,
    breaklines=true,
    frame=none,
    numbers=none,
    basicstyle=\footnotesize\ttfamily,
    keywordstyle=\bfseries\color{green!40!black},
    commentstyle=\itshape\color{purple!40!black},
    identifierstyle=\color{blue},
}

\bibliography{literature}

% DOCUMENT STARTS HERE

\begin{document}

\setlength{\headheight}{50pt}

% Header and TUM Logo
\fancypagestyle{titlepage} {
    \fancyhf{}
    \fancyhead [R] {
        \includegraphics[height=1.0cm]{TUM_Logo.png}
    }
    \fancyhead [C] {
        \large{Software Lab} \\ \large{Winter Term 2022/23}
    }
    \headsep18mm
}

% Title and Authors
\begin{center}
    \large{\textbf{Implementation of High Performance Marching Cubes Library} }\\[11pt]
\end{center}
\begin{center}
    Ivan Maximenko*,
    Qining Tang*,
    Rezha Tanuharja*\\[5pt]
\end{center}

% Footer
\long\def\symbolfootnote[#1]#2{
    \begingroup
        \def\thefootnote{\fnsymbol{footnote}}\footnote[#1]{#2}
    \endgroup
}

{
    \let\thefootnote\relax\footnotetext {
        * \textit{Technische Universit\"at M\"unchen}
    }
}




\thispagestyle{titlepage} 
\pagestyle{fancy}
\fancyhf{}
\renewcommand{\headrulewidth}{0.0pt}
\renewcommand{\footrulewidth}{0.0pt}
\fancyfoot[C]{\thepage}
\pagenumbering{arabic} 

% ABSTRACT
\section*{Motivation and Reasonings}
\begin{singlespace}
Why NI-RPCE? 
\vspace{10pt} 

In recent comparison, non - intrusive surrogate model based on rational of polynomial chaos expansions displayed promising efficiency and effectiveness to propagate uncertainties for linear structural systems. 
The main advantage of the model is its ability to decouple degree of freedoms, which enable parallel model training (once training data is available) and evaluation. 
\vspace{10pt} 

What to improve? 
\vspace{10pt} 

In the aforementioned comparison, majority of computation time is spent on training data generation.
In addition, the computation time associated with this process has the highest growth rate with respect to the structural system's complexity (size). 
There are two possible improvements: reduce the number of required training data or reduce computation time associated with generation and evaluation of the deterministic model (by performing model order reduction and/or domain decomposition). 
\vspace{10pt} 

Why deterministic model order reduction and domain decomposition?
\vspace{10pt} 

Time to generate training data is proportional to the number of data sets (linear). 
On the other hand, time to generate and evaluate deterministic model has complexity $O\left(n^{m}\right)$, with $m\geq 3$. 
Therefore, it is presumed that bigger time reduction can be achieved through model order reduction (size reduction) and domain decomposition (size reduction and parallel computing). 
\vspace{10pt} 

Why Hurty / Craig - Bampton (HCB) domain decomposition method? 
\vspace{10pt} 

This domain decomposition method is a proven and popular method in structural dynamics.
Consequently, literatures and resources on the method are plenty and widely available including multiple improvements to tackle disadvantages of the original method e.g. Schur complement for efficient parallelisation, interface mode reduction for further model order reduction, dual method to avoid locking caused by low number of modes, quasi-static component modes to improve mid- and high-frequency performance, etc..
\vspace{10pt} 

Further potential improvement: 
\vspace{10pt} 

Approximation of random eigenvectors and eigenvalues of the reduced model via perturbation method. 
HCB use eigenmode truncation to reduce model's order, therefore this approximation integrates naturally with the method. 
It can save significant time since repeated eigenvectors and eigenvalues computations can be avoided when generating training data; only deterministic ones are computed while the random ones are approximated. 
However, this potentially reduces accuracy as perturbation method is known to have poor performance in the presence of high uncertainties. 
\end{singlespace}

% KEYWORDS
% \paragraph*{Keywords:} this, that.


% SECTIONS
\newpage
\subfile{sections/DomainDecomposition} 
% \input{./sections/00_introduction}
% \input{./sections/01_sec1}
% \input{./sections/02_sec2}
%\input{./sections/03_sec3}
%\input{./sections/04_sec4}
%\input{./sections/05_sec5}
% \input{./sections/99_conclusions}

% APPENDIX
%\appendix
%\input{./sections/0inf_appendix}

% PREFERENCES
%\renewcommand*\refname{\normalsize{REFERENCES}}
%\nocite{*} %Even non-cited BibTeX-Entries will be shown.
\newpage 
\printbibliography

\end{document}
