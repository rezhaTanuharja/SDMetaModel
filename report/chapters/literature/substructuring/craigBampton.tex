% -----------------------------------------------------------------------------
% Master thesis in the study program computational mechanics
%
% B.Sc. Rezha Adrian Tanuharja - 03751261
% M.Sc. Felix Schneider (supervisor)
%
% chapters/literature/substructuring/craigBampton.tex
% Last edited 03 November 2023
% -----------------------------------------------------------------------------

\subsection{Craig-Bampton Method}
\label{ssec: CB method}

The CB method uses a transformation to a modal space.
Two modes are defined: normal and constraint modes.
{Normal modes are the component's undamped eigenmodes with all boundary DOFs constrained, i.e., the solutions to the generalized eigenproblem}%
\begin{equation}
    \mathbf{0}
    =
    \left(\mathbf{K}_{ii}-\lambda\mathbf{M}_{ii}\right)\mathbf{\Phi}_{n}.
\end{equation}
Model order reduction is possible by using only a subset of the normal modes, referred to as the dominant modes. 
The tall matrix $\mathbf{\Phi}_{d}$ stores these dominant modes.
On the other hand, each constraint mode is the substructure's static displacement when subjected to a unit displacement of a boundary DOF while the rest of the boundary DOFs are constrained.
{By this definition, the constraint modes satisfy the static equation}%
\begin{equation}
    \mathbf{0}
    =
    \mathbf{K}_{ii} \mathbf{\Phi}_{c} +
    \mathbf{K}_{ib} \mathbf{I}_{b}
    \iff
    \mathbf{\Phi}_{c}
    =
    -\mathbf{K}_{ii}^{-1}\mathbf{K}_{ib}.
\end{equation}
{The transformation $\mathbf{U}=\mathbf{T}\mathbf{X}$ relates the component's initial and modal coordinates:}%
\begin{equation}
    \begin{pmatrix}
        \mathbf{U}_{i} \\
        \mathbf{U}_{b}
    \end{pmatrix}
    =
    \begin{bmatrix}
        \;\;\mathbf{\Phi}_{d} & \;\;\mathbf{\Phi}_{c} \\
        \mathbf{0} & \mathbf{I}
    \end{bmatrix}
    \begin{pmatrix}
        \mathbf{X}_{d} \\
        \mathbf{X}_{c}
    \end{pmatrix}.
    \label{cb_transformation}
\end{equation}
Equation \eqref{cb_transformation} implies that $\mathbf{U}_{b}=\mathbf{X}_{c}$ and the two terms are interchangeable.
Algorithm \ref{alg: CB method} summarizes the procedure to compute the transformation matrix.
% -----------------------------------------------------------------------------
% Master thesis in the study program computational mechanics
%
% B.Sc. Rezha Adrian Tanuharja - 03751261
% M.Sc. Felix Schneider (supervisor)
%
% algorithm/craigBampton.tex
% Last edited 03 November 2023
% -----------------------------------------------------------------------------

\begin{center}
\begin{algorithm}[H]
    \label{alg: CB method}
    \SetKwInOut{Input}{Input}
    \SetKwInOut{Output}{Output\,}
    \vspace{1.0em}%
    \Input{%
        $\mathbf{M}$ is the component's mass matrix \newline
        $\mathbf{K}\:$ is the component's stiffness matrix \newline
        $s_{i}\:\:$ is indices of the internal DOFs \newline
        $s_{b}\:\:$ is indices of the boundary DOFs
    }
    \vspace{1.0em}%
    \Output{%
        $\mathbf{T}$ is the transformation matrix
    }
    \vspace{1.0em}%
    \craigBampton{%
        $\mathbf{M}$, $\mathbf{K}$, $s_{i}$, $s_{b}$
    } \Begin{
        \vspace{1.0em}%
        \tcp{Compute normal modes and select dominant modes}
        $\mathbf{\Phi}_{n} \longleftarrow$
        \genEigSol{$
            \mathbf{M}\left[s_{i},s_{i}\right], 
            \mathbf{K}\left[s_{i},s_{i}\right]
        $} \\
        $\mathbf{\Phi}_{d} \longleftarrow$
        \modeSelector{$
            \mathbf{\Phi}_{n}
        $} \\
        \vspace{1.0em}%
        \tcp{Compute constraint modes}
        $
            \mathbf{\Phi}_{b} \longleftarrow
            -\left(\mathbf{K}\left[s_{i},s_{i}\right]\right)^{-1}
            \mathbf{K}\left[s_{i},s_{b}\right]
        $ \\
        \vspace{1.0em}%
        \tcp{Assemble transformation matrix}
        $
            \mathbf{T} \longleftarrow
            \left[
                \begin{bmatrix}
                    \mathbf{\Phi}_{d} &
                    \mathbf{\Phi}_{c}
                \end{bmatrix}
                \begin{bmatrix}
                    \phantom{x}\mathbf{0}\phantom{.} &
                    \mathbf{I}\phantom{x}
                \end{bmatrix}
            \right]
        $ \\
        \vspace{1.0em}% 
        \Return{$\mathbf{T}$} \\
        \vspace{1.0em}%
   }
    \vspace{1.0em}%
  \caption{Craig-Bampton Transformation Matrix's Computation}
\end{algorithm}
\end{center}
{Following the transformations $\overline{\mathbf{D}} = \mathbf{T}^{T}\mathbf{D}\mathbf{T}$ and $\overline{\mathbf{F}} = \mathbf{T}^{T}\mathbf{F}$, the reduced component's dynamic equation of motion in the frequency domain is}%
\begin{equation}
    \begin{bmatrix}
        \overline{\mathbf{D}}_{dd} & \overline{\mathbf{D}}_{dc} \\
        \overline{\mathbf{D}}_{cd} & \overline{\mathbf{D}}_{cc}
    \end{bmatrix}
    \begin{pmatrix}
        \mathbf{X}_{d} \\
        \mathbf{U}_{b}
    \end{pmatrix}
    =
    \begin{pmatrix}
        \overline{\mathbf{F}}_{d} \\
        \overline{\mathbf{F}}_{c}
    \end{pmatrix}
    +
    \begin{pmatrix}
        \mathbf{0} \\
        \mathbf{G}_{b}
    \end{pmatrix}.
    \label{modal_eq_of_motion}
\end{equation}
Between \eqref{initial_eq_of_motion} and \eqref{modal_eq_of_motion}, the transformation, including the model order reduction, does not alter the boundary DOFs or the coupling force, i.e., it does not affect compatibility or equilibrium conditions among components.
As a result, the CB method enables an independent reduction of each component, leading to multiple levels of fidelity among the components.