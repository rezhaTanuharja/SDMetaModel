% -----------------------------------------------------------------------------
% Master thesis in the study program computational mechanics
%
% B.Sc. Rezha Adrian Tanuharja - 03751261
% M.Sc. Felix Schneider (supervisor)
%
% chapters/literature/stochastic/RPCE.tex
% Last edited 03 November 2023
% -----------------------------------------------------------------------------

\subsection{Rational PCE Model}
\label{ssec: RPCE}

PCE requires a large number of basis functions to accurately represent the highly nonlinear behavior of FRFs near the structure's eigenfrequencies.
As the subsection above mentions, the large number of basis functions leads to a large number of required samples, thus increasing the overall computational cost.
{An alternative is approximating FRFs with ratios of PCEs:}%
\begin{equation}
    H_{pq} \left( \omega, \mathbf{\Xi} \right)
    \approx
    \frac{
        \sum_{k\in S_{u}}{
            u_{pqk}\left(\omega\right)
            \cdot
            \Psi_{k} \left( \mathbf{\Xi} \right)
        }
    }{
        \sum_{l\in S_{v}}{
            v_{pql}\left(\omega\right)
            \cdot
            \Psi_{l} \left( \mathbf{\Xi} \right)
        }
    }
    \quad
    S_{u} = \left\{1, ..., N_{u}\right\},
    \;
    S_{v} = \left\{1, ..., N_{v}\right\}
    .
    \label{RPCE_approx}
\end{equation}
The mean square error is typically the prime candidate to be an objective function in a linear regression problem.
However, the rational form of RPCE makes finding its minima less trivial.
{In \cite{schneider2020polynomial}, the authors define an alternative objective function for the minimization problem:}%
\begin{equation}
    \left\{
        \mathbf{u}_{pq},
        \mathbf{v}_{pq}
    \right\}
    =
    \underset{
        \left\{
            \hat{\mathbf{u}},
            \hat{\mathbf{v}}
        \right\}
        \in 
        \mathbb{C}^{N_{u}+N_{v}}
    }{\arg\min}
    \;
    \frac{1}{N}
    \sum_{i=0}^{N}
    \left\|
        H_{pq} \left( \mathbf{\Xi}_{i} \right)
        \cdot
        \sum_{l\in S_{v}}{
            \hat{v}_{l} \Psi_{l} \left( \mathbf{\Xi}_{i} \right)
        }
        -
        \sum_{k\in S_{u}}{
            \hat{u}_{k} \Psi_{k} \left( \mathbf{\Xi}_{i} \right)
        }
    \right\|_{2}^{2},
\end{equation}
where the dependency on frequency is intentionally missing to keep it concise.
{Subsequently, they show that the solution satisfies}%
\begin{equation}
    \begin{bmatrix}
        \mathbf{\Psi}_{U}^{T}
        \mathbf{\Psi}_{U}
        &
        -\mathbf{\Psi}_{U}^{T}
        \mathbf{M}
        \mathbf{\Psi}_{V}
        \\
        -\mathbf{\Psi}_{V}^{T}
        \mathbf{M}^{H}
        \mathbf{\Psi}_{U}
        &
        \mathbf{\Psi}_{V}^{T}
        \mathbf{M}^{H}
        \mathbf{M}
        \mathbf{\Psi}_{V}
    \end{bmatrix}
    \begin{pmatrix}
        \mathbf{u}_{pq} \\
        \mathbf{v}_{pq}
    \end{pmatrix}
    =
    \begin{pmatrix}
        \mathbf{0} \\
        \mathbf{0}
    \end{pmatrix},
    \label{SVD problem}
\end{equation}
{where}%
\begin{equation}
    \left[\mathbf{\Psi}_{U}\right]_{ij}
    =
    \Psi_{j} \left(\mathbf{\Xi}_{i}\right)
    \quad
    \forall
    \quad
    (i,j)\in\left\{1, ..., N\right\} \otimes
    \left\{1, ..., N_{u}\right\},
\end{equation}
\vspace{-3.0em}
\begin{equation}
    \left[\mathbf{\Psi}_{V}\right]_{ij}
    =
    \Psi_{j} \left(\mathbf{\Xi}_{i}\right)
    \quad
    \forall
    \quad
    (i,j)\in\left\{1, ..., N\right\} \otimes
    \left\{1, ..., N_{v}\right\},
\end{equation}
{and}%
\begin{equation}
    \mathbf{M}
    =
    \text{diag} \left(
        H_{pq} \left( \mathbf{\Xi}_{1} \right),
        H_{pq} \left( \mathbf{\Xi}_{2} \right),
        ..., 
        H_{pq} \left( \mathbf{\Xi}_{N} \right)
    \right).
\end{equation}