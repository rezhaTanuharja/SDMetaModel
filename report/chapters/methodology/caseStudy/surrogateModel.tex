% -----------------------------------------------------------------------------
% Master thesis in the study program computational mechanics
%
% B.Sc. Rezha Adrian Tanuharja - 03751261
% M.Sc. Felix Schneider (supervisor)
%
% chapters/methodology/caseStudy/surrogateModel.tex
% Last edited 03 November 2023
% -----------------------------------------------------------------------------

\subsection{Sparse NI-RPCE Model}
\label{ssec: surrogate model}

This study uses RPCE models with products of probabilist Hermite polynomials as the basis functions, as in \eqref{prob Hermite products}.
The numerator is a PCE with an order of $5$ while the denominator is a PCE with an order of $6$.
Therefore, the basis functions in the numerator and denominator satisfy
\begin{equation}
    \Psi_{k}\left(\mathbf{\Xi}\right)
    =
    {He}_{r_{k}} \left(\xi_{1}\right)
    \cdot
    {He}_{s_{k}} \left(\xi_{2}\right)
    \phantom{x}
    \forall 
    \phantom{x}
    r_{k}, s_{k} \in \mathbb{N}, 
    \phantom{x}
    r_{k} + s_{k} \leq 5
\end{equation}
and
\begin{equation}
    \Psi_{l}\left(\mathbf{\Xi}\right)
    =
    {He}_{r_{l}} \left(\xi_{1}\right)
    \cdot
    {He}_{s_{l}} \left(\xi_{2}\right)
    \phantom{x}
    \forall 
    \phantom{x}
    r_{l}, s_{l} \in \mathbb{N}, 
    \phantom{x}
    r_{l} + s_{l} \leq 6,
\end{equation}
respectively.
The author obtains the coefficient in \eqref{RPCE_approx} by solving \eqref{SVD problem}.
There are $49$ basis functions in the numerator and denominator of the RPCE models.
The author reduces the number of basis functions by following the algorithms \ref{alg: sparse RPCE numerator} and \ref{alg: sparse RPCE denominator} with $\epsilon_{0}=1.0$.