% -----------------------------------------------------------------------------
% Master thesis in the study program computational mechanics
%
% B.Sc. Rezha Adrian Tanuharja - 03751261
% M.Sc. Felix Schneider (supervisor)
%
% decorators/abstract.tex
% Last edited 03 November 2023
% -----------------------------------------------------------------------------

\addchap{Abstract}
\label{cha:abtract} 

This thesis introduces a novel framework for uncertainty quantification in linear structural dynamics.
The framework uses the well-established Craig-Bampton (CB) method to generate the training data for a recently developed non-intrusive rational polynomial chaos expansion (NI-RPCE) model.
A modification to the CB method takes place, reducing its computational cost in a Monte Carlo simulation (MCS) for models with varying input parameters.
In addition, the author develops a new approach to building a sparse NI-RPCE model, thus reducing the required size of the training data.
As a result, the proposed framework provides an efficient and scalable means to tackle large and complex structures.
The scope covers time-invariant structures with parametric uncertainties.
The thesis includes a case study to compare the new framework with a classic uncertainty quantification approach: the direct MCS.


Keywords: 
\begin{itemize}
 \item Craig-Bampton Method
 \item Linear Structural Dynamics
 \item Non-Intrusive Rational Polynomial Chaos Expansion
 \item Sparse Surrogate Model
 \item Uncertainty Quantification
\end{itemize}