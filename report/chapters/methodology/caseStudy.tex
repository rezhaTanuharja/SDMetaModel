% -----------------------------------------------------------------------------
% Master thesis in the study program computational mechanics
%
% B.Sc. Rezha Adrian Tanuharja - 03751261
% M.Sc. Felix Schneider (supervisor)
%
% chapters/methodology/caseStudy.tex
% Last edited 03 November 2023
% -----------------------------------------------------------------------------

\section{Case Study}
\label{sec: case study}

This study demonstrates the practical application of the new framework using a simple yet relevant example: a plate clamped at both of its ends.
The choice of case is deliberate, as it is sufficiently large for the substructuring methods while still maintaining the feasibility of direct MC simulations.
Figure \ref{fig: case study} illustrates the plate and its dimensions.

% -----------------------------------------------------------------------------
% Master thesis in the study program computational mechanics
%
% B.Sc. Rezha Adrian Tanuharja - 03751261
% M.Sc. Felix Schneider (supervisor)
%
% images/caseStudy.tex
% Last edited 03 November 2023
% -----------------------------------------------------------------------------

\begin{figure}[H]
    \centering

    \begin{tikzpicture}[
        scale=0.35, 
        %
        % Isometric-esque appearance
        x={( 0.6cm,-0.2cm)}, 
        y={( 0.5cm, 0.3cm)}, 
        z={( 0.0cm, 1.0cm)}, 
        %
        draw opacity=0.7
    ]

        % Define plate parameters
        % maxes are lengths - one step because of plotting
        \def\xmin{0} \def\xmax{47.0}
        \def\ymin{0} \def\ymax{22.8}
        % the dimension of each rectangular element
        \def\xstep{1.0}
        \def\ystep{1.2}

        % Define clamping wall parameters
        \def\wallh{1.2}
        \def\wallt{1.0}

        % Draw the side left clamping wall 
        \draw[pattern=north west lines] 
            (   \xmin, \ymin, -\wallh) -- 
            (   \xmin, \ymin,  \wallh) -- 
            ( -\wallt, \ymin,  \wallh) -- 
            ( -\wallt, \ymin, -\wallh) -- 
            cycle;

        % Draw the top left clamping wall
        \draw[pattern=north west lines] 
            (   \xmin, \ymin       ,  \wallh) -- 
            (   \xmin, \ymax+\ystep,  \wallh) -- 
            ( -\wallt, \ymax+\ystep,  \wallh) -- 
            ( -\wallt, \ymin       ,  \wallh) -- 
            cycle;

        % Draw the front left clamping wall
        \draw[fill=white]
            (   \xmin, \ymin       , -\wallh) -- 
            (   \xmin, \ymax+\ystep, -\wallh) --
            (   \xmin, \ymax+\ystep,  \wallh) --
            (   \xmin, \ymin       ,  \wallh) -- 
            cycle; 
    
        % Draw the plate
        \draw[fill=white] 
            ( \xmin       , \ymin       , 0) -- 
            ( \xmax+\xstep, \ymin       , 0) -- 
            ( \xmax+\xstep, \ymax+\ystep, 0) -- 
            ( \xmin       , \ymax+\ystep, 0) -- 
            cycle;

        % Draw annotations
        \draw[thin]
            ( \xmin, \ymax+  \ystep, 0 ) --
            ( \xmin, \ymax+4*\ystep, 0);

        \draw[thin]
            ( \xmax+\xstep, \ymax+  \ystep, 0 ) --
            ( \xmax+\xstep, \ymax+4*\ystep, 0);

        \draw[latex-latex]
            ( \xmin       , \ymax+3*\ystep, 0) --
            node[sloped, anchor=center, above] {$8.0\text{ m}$}
            ( \xmax+\xstep, \ymax+3*\ystep, 0);

        \draw[latex-latex]
            ( \xmax-5*\xstep, \ymin, 0 ) --
            node[sloped, anchor=center, above] {$4.0\text{ m}$}
            ( \xmax-5*\xstep, \ymax+\ystep, 0 );

        % Put the back right clamping wall in front of the plate
        \draw[fill=white]
            (   \xmax+\wallt+\xstep, \ymin       , -\wallh) -- 
            (   \xmax+\wallt+\xstep, \ymax+\ystep, -\wallh) --
            (   \xmax+\wallt+\xstep, \ymax+\ystep,  \wallh) --
            (   \xmax+\wallt+\xstep, \ymin       ,  \wallh) -- 
            cycle; 

        % Draw the back right clamping wall
        \draw[pattern=north east lines]
            (   \xmax+\wallt+\xstep, \ymin       , -\wallh) -- 
            (   \xmax+\wallt+\xstep, \ymax+\ystep, -\wallh) --
            (   \xmax+\wallt+\xstep, \ymax+\ystep,  \wallh) --
            (   \xmax+\wallt+\xstep, \ymin       ,  \wallh) -- 
            cycle; 

        % Put the side right clamping wall in front of the plate
        \draw[fill=white] 
            ( \xmax+\xstep       , \ymin, -\wallh) -- 
            ( \xmax+\xstep       , \ymin,  \wallh) -- 
            ( \xmax+\xstep+\wallt, \ymin,  \wallh) -- 
            ( \xmax+\xstep+\wallt, \ymin, -\wallh) -- 
            cycle;
        
        % Draw the side right clamping wall
        \draw[pattern=north west lines] 
            ( \xmax+\xstep       , \ymin, -\wallh) -- 
            ( \xmax+\xstep       , \ymin,  \wallh) -- 
            ( \xmax+\xstep+\wallt, \ymin,  \wallh) -- 
            ( \xmax+\xstep+\wallt, \ymin, -\wallh) -- 
            cycle;
        
        % Put the top right clamping wall in front of the plate
        \draw[fill=white] 
            ( \xmax+\xstep       , \ymin       ,  \wallh) -- 
            ( \xmax+\xstep       , \ymax+\ystep,  \wallh) -- 
            ( \xmax+\xstep+\wallt, \ymax+\ystep,  \wallh) -- 
            ( \xmax+\xstep+\wallt, \ymin       ,  \wallh) -- 
            cycle;

        % Draw the top right clamping wall
        \draw[pattern=north west lines] 
            ( \xmax+\xstep       , \ymin       ,  \wallh) -- 
            ( \xmax+\xstep       , \ymax+\ystep,  \wallh) -- 
            ( \xmax+\xstep+\wallt, \ymax+\ystep,  \wallh) -- 
            ( \xmax+\xstep+\wallt, \ymin       ,  \wallh) -- 
            cycle;

        % Draw coordinate axis
        \draw[thick, ->]
            ( \xmin, \ymin-6*\ystep, 0) --
            ( \xmin, \ymin-6*\ystep, -1.5*\xstep)
            node [below] {$u$};

        \draw[thick, ->]
            ( \xmin         , \ymin-6*\ystep, 0) --
            ( \xmin+2*\xstep, \ymin-6*\ystep, 0)
            node [right, xshift=-1.5*\xstep, yshift=-1.5*\ystep] {$x$};

        \draw[thick, ->]
            ( \xmin, \ymin-6*\ystep, 0) --
            ( \xmin, \ymin-4*\ystep, 0)
            node [above, xshift=5.5*\xstep, yshift=-2.5*\ystep] {$y$};

    \end{tikzpicture}
    \caption{Case Study: A Plate Clamped at Both Ends}
    \label{fig: case study}
\end{figure}

Table \ref{table: Plate Parameter} lists the relevant plate's parameters.
The plate's density and thickness are uncertain and modeled by random variables.
The parameters $\xi_{1}$ and $\xi_{2}$ are statistically independent standard normal random variables.

\begin{table}[H]
    \setlength{\extrarowheight}{2pt}
    \centering
    \begin{tabular}{
        L{0.32\textwidth}rccC{0.25\textwidth}
    }
        \hline
        Parameter & \multicolumn{3}{c}{Value} & Unit \\
        \hline
        Thickness &
            $(1.0+0.2\xi_{1})$ & $\times$ & $0.01$ &
            \einheit{m} \\
        Density & 
            $(1.0+0.2\xi_{2})$ & $\times$ & $7850.0$ &
            \einheit{kg/{m}^{3}} \\
        Shear modulus & 
            \multicolumn{3}{c}{$79.3$} &
            \einheit{GPa} \\
        Shear correction factor & 
            \multicolumn{3}{c}{$0.83$} &
            -- \\
        Poisson ratio & 
            \multicolumn{3}{c}{$0.3$} &
            -- \\
        \hline
    \end{tabular}
    \caption{Plate's Parameters in The Case Study}
    \label{table: Plate Parameter}
\end{table}

Two discrete models approximate the plate using the FE method with Kirchoff plate theory: the complete plate model and the substructured plate model.
The proposed framework uses the latter to generate training data for the NI-RPCE models, which will approximate the complete plate model.
On the other hand, the FRFs from direct MCS using the complete plate model serve as the baselines for comparison.

% -----------------------------------------------------------------------------
% Master thesis in the study program computational mechanics
%
% B.Sc. Rezha Adrian Tanuharja - 03751261
% M.Sc. Felix Schneider (supervisor)
%
% images/completeModel.tex
% Last edited 03 November 2023
% -----------------------------------------------------------------------------

\begin{figure}[H]
    \centering

    \begin{tikzpicture}[
        scale=0.35, 
        %
        % Isometric-esque appearance
        x={( 0.6cm,-0.2cm)}, 
        y={( 0.5cm, 0.3cm)}, 
        z={( 0.0cm, 1.0cm)}, 
        %
        draw opacity=0.7
    ]

        % Define grid parameters
        % maxes are lengths - one step because of plotting
        \def\xmin{0} \def\xmax{47.0}
        \def\ymin{0} \def\ymax{22.8}
        % the dimension of each rectangular element
        \def\xstep{1.0}
        \def\ystep{1.2}

        % Define clamping wall parameters
        \def\wallh{1.2}
        \def\wallt{1.0}

        % Define markers' diameter
        \def\noder{0.5}

        % Draw the side left clamping wall 
        \draw[pattern=north west lines] 
            (   \xmin, \ymin, -\wallh) -- 
            (   \xmin, \ymin,  \wallh) -- 
            ( -\wallt, \ymin,  \wallh) -- 
            ( -\wallt, \ymin, -\wallh) -- 
            cycle;

        % Draw the top left clamping wall
        \draw[pattern=north west lines] 
            (   \xmin, \ymin       ,  \wallh) -- 
            (   \xmin, \ymax+\ystep,  \wallh) -- 
            ( -\wallt, \ymax+\ystep,  \wallh) -- 
            ( -\wallt, \ymin       ,  \wallh) -- 
            cycle;

        % Draw the front left clamping wall
        \draw[fill=white]
            (   \xmin, \ymin       , -\wallh) -- 
            (   \xmin, \ymax+\ystep, -\wallh) --
            (   \xmin, \ymax+\ystep,  \wallh) --
            (   \xmin, \ymin       ,  \wallh) -- 
            cycle; 
    
        % Draw grid
        \foreach \x in {\xmin,\xstep,...,\xmax}
            \foreach \y in {\ymin,\ystep,...,\ymax}
                \draw[fill=white] 
                      ( \x   , \y   , 0) -- 
                    ++( \xstep, 0    , 0) -- 
                    ++( 0    , \ystep, 0) -- 
                    ++(-\xstep, 0    , 0) -- 
                    cycle;

        % Put the back right clamping wall in front of the grid
        \draw[fill=white]
            (   \xmax+\wallt+\xstep, \ymin       , -\wallh) -- 
            (   \xmax+\wallt+\xstep, \ymax+\ystep, -\wallh) --
            (   \xmax+\wallt+\xstep, \ymax+\ystep,  \wallh) --
            (   \xmax+\wallt+\xstep, \ymin       ,  \wallh) -- 
            cycle; 

        % Draw the back right clamping wall
        \draw[pattern=north east lines]
            (   \xmax+\wallt+\xstep, \ymin       , -\wallh) -- 
            (   \xmax+\wallt+\xstep, \ymax+\ystep, -\wallh) --
            (   \xmax+\wallt+\xstep, \ymax+\ystep,  \wallh) --
            (   \xmax+\wallt+\xstep, \ymin       ,  \wallh) -- 
            cycle; 

        % Put the side right clamping wall in front of the grid
        \draw[fill=white] 
            ( \xmax+\xstep       , \ymin, -\wallh) -- 
            ( \xmax+\xstep       , \ymin,  \wallh) -- 
            ( \xmax+\xstep+\wallt, \ymin,  \wallh) -- 
            ( \xmax+\xstep+\wallt, \ymin, -\wallh) -- 
            cycle;
        
        % Draw the side right clamping wall
        \draw[pattern=north west lines] 
            ( \xmax+\xstep       , \ymin, -\wallh) -- 
            ( \xmax+\xstep       , \ymin,  \wallh) -- 
            ( \xmax+\xstep+\wallt, \ymin,  \wallh) -- 
            ( \xmax+\xstep+\wallt, \ymin, -\wallh) -- 
            cycle;
        
        % Put the top right clamping wall in front of the grid
        \draw[fill=white] 
            ( \xmax+\xstep       , \ymin       ,  \wallh) -- 
            ( \xmax+\xstep       , \ymax+\ystep,  \wallh) -- 
            ( \xmax+\xstep+\wallt, \ymax+\ystep,  \wallh) -- 
            ( \xmax+\xstep+\wallt, \ymin       ,  \wallh) -- 
            cycle;

        % Draw the top right clamping wall
        \draw[pattern=north west lines] 
            ( \xmax+\xstep       , \ymin       ,  \wallh) -- 
            ( \xmax+\xstep       , \ymax+\ystep,  \wallh) -- 
            ( \xmax+\xstep+\wallt, \ymax+\ystep,  \wallh) -- 
            ( \xmax+\xstep+\wallt, \ymin       ,  \wallh) -- 
            cycle;

        % Draw a marker for the node B
        \draw[fill=red] 
            ( 8*\xstep, 9.5*\ystep ) ellipse (0.3 and 0.3);
        % Draw a line pointing at the marker for the node B
        \draw[thick]
            (  8*\xstep, 9.5*\ystep,        0 ) --
            ( -1*\xstep, 9.5*\ystep, 2*\wallh );
        % Add label B
        \node at
            ( -2*\xstep, 9.5*\ystep, 2*\wallh+0.6*\xstep ) {B};

        % Draw a marker for the node A
        \draw[fill=red] 
            ( 32*\xstep, 20*\ystep ) ellipse (0.3 and 0.3);
        % Draw a line pointing at the two markers for the node A
        \draw[thick]
            ( 32*\xstep         , 20*\ystep       , 0               ) --
            ( 32*\xstep+7*\xstep, 20*\ystep+\ystep, \wallh+3*\xstep );
        % Add label A
        \node at (32*\xstep+8*\xstep,20*\ystep+\ystep, \wallh+4*\xstep) {A};

    \end{tikzpicture}
    \caption{Meshed Complete Plate Model with Locations of Nodes A and B}
    \label{fig: complete model}
\end{figure}
% -----------------------------------------------------------------------------
% Master thesis in the study program computational mechanics
%
% B.Sc. Rezha Adrian Tanuharja - 03751261
% M.Sc. Felix Schneider (supervisor)
%
% chapters/methodology/caseStudy/substructuredModel.tex
% Last edited 03 November 2023
% -----------------------------------------------------------------------------

\subsection{Substructured Plate Model}
\label{ssec: substructured model}

The substructured plate model partitions the plate into three components along the $x$ axis.
The components have different numbers of DOFs and different modes because their lengths are not equal.
This choice is intentional, since in practical applications, components rarely have the same number of DOFs and the same modes.
Table \ref{table: components' parameters} provides the meshes' parameters for the three components.

\begin{table}[H]
    \setlength{\extrarowheight}{2pt}
    \centering
    \begin{tabular}{
        L{0.3\textwidth} C{0.15\textwidth} C{0.15\textwidth} C{0.15\textwidth}
    }
    \hline
    Parameter & Left & Middle & Right \\
    \hline
    Number of element (x) &
    16 & 14 & 18 \\
    Number of element (y) &
    20 & 20 & 20 \\
    & & & \\
    Number of Nodes &
    1353 & 1189 & 1517 \\
    Number of DOFs &
    4059 & 3567 & 4551 \\
    \hline
    \end{tabular}
    \caption{Substructured Plate Model's Meshes' Parameters}
    \label{table: components' parameters}
\end{table}

The author selects the meshes' parameters such that the elements' sizes are uniform and equal to the elements' sizes in the complete plate model.
This is to avoid numerical artifacts in the comparison caused by different elements' sizes between the two models.
Figure \ref{fig: substructured model} illustrates the substructured plate model and the two nodes.

% -----------------------------------------------------------------------------
% Master thesis in the study program computational mechanics
%
% B.Sc. Rezha Adrian Tanuharja - 03751261
% M.Sc. Felix Schneider (supervisor)
%
% chapters/methodology/caseStudy/substructuredModel.tex
% Last edited 03 November 2023
% -----------------------------------------------------------------------------

\subsection{Substructured Plate Model}
\label{ssec: substructured model}

The substructured plate model partitions the plate into three components along the $x$ axis.
The components have different numbers of DOFs and different modes because their lengths are not equal.
This choice is intentional, since in practical applications, components rarely have the same number of DOFs and the same modes.
Table \ref{table: components' parameters} provides the meshes' parameters for the three components.

\begin{table}[H]
    \setlength{\extrarowheight}{2pt}
    \centering
    \begin{tabular}{
        L{0.3\textwidth} C{0.15\textwidth} C{0.15\textwidth} C{0.15\textwidth}
    }
    \hline
    Parameter & Left & Middle & Right \\
    \hline
    Number of element (x) &
    16 & 14 & 18 \\
    Number of element (y) &
    20 & 20 & 20 \\
    & & & \\
    Number of Nodes &
    1353 & 1189 & 1517 \\
    Number of DOFs &
    4059 & 3567 & 4551 \\
    \hline
    \end{tabular}
    \caption{Substructured Plate Model's Meshes' Parameters}
    \label{table: components' parameters}
\end{table}

The author selects the meshes' parameters such that the elements' sizes are uniform and equal to the elements' sizes in the complete plate model.
This is to avoid numerical artifacts in the comparison caused by different elements' sizes between the two models.
Figure \ref{fig: substructured model} illustrates the substructured plate model and the two nodes.

% -----------------------------------------------------------------------------
% Master thesis in the study program computational mechanics
%
% B.Sc. Rezha Adrian Tanuharja - 03751261
% M.Sc. Felix Schneider (supervisor)
%
% chapters/methodology/caseStudy/substructuredModel.tex
% Last edited 03 November 2023
% -----------------------------------------------------------------------------

\subsection{Substructured Plate Model}
\label{ssec: substructured model}

The substructured plate model partitions the plate into three components along the $x$ axis.
The components have different numbers of DOFs and different modes because their lengths are not equal.
This choice is intentional, since in practical applications, components rarely have the same number of DOFs and the same modes.
Table \ref{table: components' parameters} provides the meshes' parameters for the three components.

\begin{table}[H]
    \setlength{\extrarowheight}{2pt}
    \centering
    \begin{tabular}{
        L{0.3\textwidth} C{0.15\textwidth} C{0.15\textwidth} C{0.15\textwidth}
    }
    \hline
    Parameter & Left & Middle & Right \\
    \hline
    Number of element (x) &
    16 & 14 & 18 \\
    Number of element (y) &
    20 & 20 & 20 \\
    & & & \\
    Number of Nodes &
    1353 & 1189 & 1517 \\
    Number of DOFs &
    4059 & 3567 & 4551 \\
    \hline
    \end{tabular}
    \caption{Substructured Plate Model's Meshes' Parameters}
    \label{table: components' parameters}
\end{table}

The author selects the meshes' parameters such that the elements' sizes are uniform and equal to the elements' sizes in the complete plate model.
This is to avoid numerical artifacts in the comparison caused by different elements' sizes between the two models.
Figure \ref{fig: substructured model} illustrates the substructured plate model and the two nodes.

\input{images/substructuredModel}
% -----------------------------------------------------------------------------
% Master thesis in the study program computational mechanics
%
% B.Sc. Rezha Adrian Tanuharja - 03751261
% M.Sc. Felix Schneider (supervisor)
%
% chapters/methodology/caseStudy/surrogateModel.tex
% Last edited 03 November 2023
% -----------------------------------------------------------------------------

\subsection{Sparse NI-RPCE Model}
\label{ssec: surrogate model}

This study uses RPCE models with products of probabilist Hermite polynomials as the basis functions, as in \eqref{prob Hermite products}.
The numerator is a PCE with an order of $5$ while the denominator is a PCE with an order of $6$.
Therefore, the basis functions in the numerator and denominator satisfy
\begin{equation}
    \Psi_{k}\left(\mathbf{\Xi}\right)
    =
    {He}_{r_{k}} \left(\xi_{1}\right)
    \cdot
    {He}_{s_{k}} \left(\xi_{2}\right)
    \phantom{x}
    \forall 
    \phantom{x}
    r_{k}, s_{k} \in \mathbb{N}, 
    \phantom{x}
    r_{k} + s_{k} \leq 5
\end{equation}
and
\begin{equation}
    \Psi_{l}\left(\mathbf{\Xi}\right)
    =
    {He}_{r_{l}} \left(\xi_{1}\right)
    \cdot
    {He}_{s_{l}} \left(\xi_{2}\right)
    \phantom{x}
    \forall 
    \phantom{x}
    r_{l}, s_{l} \in \mathbb{N}, 
    \phantom{x}
    r_{l} + s_{l} \leq 6,
\end{equation}
respectively.
The author obtains the coefficient in \eqref{RPCE_approx} by solving \eqref{SVD problem}.
There are $49$ basis functions in the numerator and denominator of the RPCE models.
The author reduces the number of basis functions by following the algorithms \ref{alg: sparse RPCE numerator} and \ref{alg: sparse RPCE denominator} with $\epsilon_{0}=1.0$.