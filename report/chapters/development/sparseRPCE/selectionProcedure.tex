% -----------------------------------------------------------------------------
% Master thesis in the study program computational mechanics
%
% B.Sc. Rezha Adrian Tanuharja - 03751261
% M.Sc. Felix Schneider (supervisor)
%
% chapters/development/sparseRPCE/selectionProcedure.tex
% Last edited 03 November 2023
% -----------------------------------------------------------------------------

\subsection{Basis Selection Procedure}
\label{ssec: selection procedure}

The procedure starts with all basis functions in the numerator and denominator.
It removes a basis function from the numerator and evaluates the approximation's quality.
If the coefficient of determination significantly declines, the basis function returns to the model.
{This study defines a significant decline in the coefficient of determination as:}%
\begin{equation}
    R^{2} - R_{-}^{2} > \epsilon_{0} \cdot \left(1-R^{2}\right),
    \label{decline criteria}
\end{equation}
where $\epsilon_{0}$ is a tuning parameter, $R^{2}$ and $R_{-}^{2}$ are the coefficient of determinations before and after the basis removal, respectively.
The author repeats these steps for all basis functions in the numerator, from the highest order to the lowest order.
Algorithm \ref{alg: sparse RPCE numerator} illustrates this process.

% -----------------------------------------------------------------------------
% Master thesis in the study program computational mechanics
%
% B.Sc. Rezha Adrian Tanuharja - 03751261
% M.Sc. Felix Schneider (supervisor)
%
% algorithm/sparseRPCENumerator.tex
% Last edited 03 November 2023
% -----------------------------------------------------------------------------

\begin{center}
\begin{algorithm}[H]
    \label{alg: sparse RPCE numerator}
    \SetKwInOut{Input}{Input}
    \SetKwInOut{Output}{Output\,}
    \vspace{1.0em}%
    \Input{%
        $\mathbf{Y}\:\:$ is the sample model outputs \newline
        $\mathbf{A}_{u}$ is the sample regressors in the numerator \newline
        $\mathbf{A}_{v}$ is the sample regressors in the denominator
    }
    \vspace{1.0em}%
    \Output{%
        $S_{u}$ is the selected numerator indices \newline
        $S_{v}$ is the selected denominator indices
    }
    \vspace{1.0em}%
    \basisSelector{%
        $\mathbf{Y}$,
        $\mathbf{A}_{u}$,
        $\mathbf{A}_{v}$
    } \Begin{
        \tcp{Initialize numerator and denominator's indices}
        $N_{u} \longleftarrow$ \len{$\mathbf{A}_{u}$} \\ 
        $N_{v} \longleftarrow$ \len{$\mathbf{A}_{v}$} \\ 
        \vspace{1.0em}%
        $S_{u} \longleftarrow \left\{1, ..., N_{u}\right\}$ \\
        $S_{v} \longleftarrow \left\{1, ..., N_{v}\right\}$ \\
        \vspace{1.0em}%
        \tcp{Initialize coefficient of determination}
        $\hat{\mathbf{Y}} \longleftarrow$
        \looRPCE{$\mathbf{Y}$, $\mathbf{A}_{u}$, $\mathbf{A}_{v}$, $S_{u}$, $S_{v}$} \\
        $R^{2} \longleftarrow$
        \detCoeff{$\mathbf{Y}, \hat{\mathbf{Y}}$}\\
        \vspace{1.0em}%
        \For{$i = N_{u}, ..., 1$}{
            \vspace{1.0em}%
            $S_{u} \setminus \{i\}$ \\
            \vspace{1.0em}%
            $\hat{\mathbf{Y}} \longleftarrow$
            \looRPCE{$\mathbf{Y}$, $\mathbf{A}_{u}$, $\mathbf{A}_{v}$, $S_{u}$, $S_{v}$} \\
            \vspace{1.0em}%
            $R_{-}^{2} \longleftarrow$
            \detCoeff{$\mathbf{Y}, \hat{\mathbf{Y}}$}\\
            \vspace{1.0em}%
            \uIf{$R^{2}-R_{-}^{2} > \epsilon_{0}\cdot(1-R^{2})$}{
                $S_{u} \longleftarrow S_{u}\cup\left\{i\right\}$
            }
            \Else{
                $R^{2} \longleftarrow R_{-}^{2}$
            }
        }
        \vspace{1.0em}%
        \tcc{Removal steps for denominator}
        \vspace{1.0em}%
        \Return{$S_{u}$, $S_{v}$} \\
        \vspace{1.0em}%
    }
    \vspace{1.0em}%
  \caption{RPCE Numerator's Basis Function Selection}
\end{algorithm}
\end{center}

After going through all of the basis functions in the numerator, the same procedure takes place for the denominator.
Algorithm \ref{alg: sparse RPCE denominator} illustrates this process.

% -----------------------------------------------------------------------------
% Master thesis in the study program computational mechanics
%
% B.Sc. Rezha Adrian Tanuharja - 03751261
% M.Sc. Felix Schneider (supervisor)
%
% algorithm/sparseRPCEDenominator.tex
% Last edited 03 November 2023
% -----------------------------------------------------------------------------

\begin{center}
\begin{algorithm}[H]
    \label{alg: sparse RPCE denominator}
    \SetKwInOut{Input}{Input}
    \SetKwInOut{Output}{Output\,}
    \vspace{1.0em}%
    \Input{%
        $\mathbf{Y}\:\:$ is the sample model outputs \newline
        $\mathbf{A}_{u}$ is the sample regressors in the numerator \newline
        $\mathbf{A}_{v}$ is the sample regressors in the denominator
    }
    \vspace{1.0em}%
    \Output{%
        $S_{u}$ is the selected numerator indices \newline
        $S_{v}$ is the selected denominator indices
    }
    \vspace{1.0em}%
    \basisSelector{%
        $\mathbf{Y}$,
        $\mathbf{A}_{u}$,
        $\mathbf{A}_{v}$
    } \Begin{
        \vspace{1.0em}%
        \vspace{1.0em}%
        $\dots$ \\
        \tcc{Removal steps for numerator}
        \vspace{1.0em}%
        \For{$i = N_{v}, ..., 1$}{
            \vspace{1.0em}%
            $S_{v} \setminus \{i\}$ \\
            \vspace{1.0em}%
            $\hat{\mathbf{Y}} \longleftarrow$
            \looRPCE{$\mathbf{Y}$, $\mathbf{A}_{u}$, $\mathbf{A}_{v}$, $S_{u}$, $S_{v}$} \\
            \vspace{1.0em}%
            $R_{-}^{2} \longleftarrow$
            \detCoeff{$\mathbf{Y}, \hat{\mathbf{Y}}$}\\
            \vspace{1.0em}%
            \uIf{$R^{2}-R_{-}^{2} > \epsilon_{0}\cdot(1-R^{2})$}{
                $S_{v} \longleftarrow S_{v}\cup\left\{i\right\}$ \\
            }
            \Else{
                $R^{2} \longleftarrow R_{-}^{2}$
            }
            \vspace{1.0em}
        }
        \vspace{1.0em}%
        \Return{$S_{u}$, $S_{v}$} \\
        \vspace{1.0em}%
    }
    \vspace{1.0em}%
  \caption{RPCE Denominator's Basis Functions Selection}
\end{algorithm}
\end{center}