% -----------------------------------------------------------------------------
% Master thesis in the study program computational mechanics
%
% B.Sc. Rezha Adrian Tanuharja - 03751261
% M.Sc. Felix Schneider (supervisor)
%
% chapters/development/uncertainCB/hybridUCB.tex
% Last edited 03 November 2023
% -----------------------------------------------------------------------------

\subsection{Hybrid Uncertain CB Method}
\label{ssec: HUCB}

The hybrid UCB method is an attempt to balance accuracy and efficiency.
It combines the naive and crude approaches.
The new approach computes the normal modes using a reference structure and reuses it for all samples.
However, it recomputes the constraint modes once for each sample in each frequency of interest.
Therefore, the method has a lower computational cost than the naive method because the computation of the normal modes only occurs once.
In addition, the method has better accuracy than the crude method because it calculates the constraint modes for each set of input parameters.

The hybrid UCB method uses a simple selection approach for the dominant modes: select a subset of eigenvalues that are closest to $\omega^{2}$ and use their associated eigenmodes as the dominant modes.
Mathematically, this is equivalent to solving the eigenproblem
\begin{equation}
    \left(
        \mathbf{K} - \omega^{2} \mathbf{M}
    \right)^{-1}
    \mathbf{M}
    \mathbf{\Phi}
    =
    \tilde{\lambda}
    \mathbf{\Phi}
\end{equation}
and use the eigenmodes associated with the largest eigenvalues' magnitudes.
% where $\tilde{\lambda}=\left(\lambda - \omega^{2}\right)^{-1}$

The hybrid UCB method uses a different definition of the constraint modes.
Each constraint mode is the substructure's dynamic displacement when subjected to a unit displacement of a boundary DOF while the rest of the boundary DOFs are constrained.
{By this definition, the constraint modes satisfy the dynamic equation}%
\begin{equation}
    \mathbf{0}
    =
    \mathbf{D}_{ii}
    \tilde{\mathbf{\Phi}}_{c}
    +
    \mathbf{D}_{ib}
    \mathbf{I}
    \iff
    \tilde{\mathbf{\Phi}}_{c}
    =
    -\mathbf{D}_{ii}^{-1}
    \mathbf{D}_{ib}.
    \label{modified_constraint_modes}
\end{equation}
{Using the new constraint modes, the enhanced transformation between the component's initial coordinates and the modal coordinates \eqref{enhanced_transformation} becomes}
\begin{equation}
    \begin{pmatrix}
        \mathbf{U}_{i} \\
        \mathbf{U}_{b}
    \end{pmatrix}
    =
    \begin{bmatrix}
        \mathbf{\Phi}_{d} &
        -\mathbf{D}_{ii}^{-1} \mathbf{D}_{ib}
        +
        \left(
            \mathbf{\Phi}_{d}
            \mathbf{\Lambda}_{d}^{-1}
            \mathbf{\Phi}_{d}^{T}
        \right)
        \left(
            \mathbf{D}_{ii}
            \tilde{\mathbf{\Phi}}_{c}
            +
            \mathbf{D}_{ib}
        \right)
        \\
        \mathbf{0}
        &
        \mathbf{I}
    \end{bmatrix}
    \begin{pmatrix}
        \mathbf{X}_{d} \\
        \mathbf{U}_{b}
    \end{pmatrix}.
    \label{modified_enhanced_transformation}
\end{equation}
{The substitution of \eqref{modified_constraint_modes} into \eqref{modified_enhanced_transformation} yields}%
\begin{equation}
    \begin{pmatrix}
        \mathbf{U}_{i} \\
        \mathbf{U}_{b}
    \end{pmatrix}
    =
    \begin{bmatrix}
        \mathbf{\Phi}_{d} &
        \tilde{\mathbf{\Phi}}_{c} \\
        \mathbf{0} & \mathbf{I}
    \end{bmatrix}
    \begin{pmatrix}
        \mathbf{X}_{d} \\
        \mathbf{U}_{b}
    \end{pmatrix},
\end{equation}
which is the transformation matrix of a normal CB method when using the modified constraint modes.
Algorithm \ref{alg: HUCB} depicts the hybrid UCB method.

% -----------------------------------------------------------------------------
% Master thesis in the study program computational mechanics
%
% B.Sc. Rezha Adrian Tanuharja - 03751261
% M.Sc. Felix Schneider (supervisor)
%
% algorithm/hybridUCB.tex
% Last edited 03 November 2023
% -----------------------------------------------------------------------------

\begin{center}
\begin{algorithm}[H]
    \label{alg: HUCB}
    \vspace{1.0em}%
    \tcp{Compute reference normal modes}
    $\mathbf{\Xi}_{0} \:\longleftarrow \mathbf{0}$ \\
    $\mathbf{\Phi}_{n} \longleftarrow$
    \genEigSol{$
        \mathbf{M}_{ii}\left(\mathbf{\Xi}_{0}\right), 
        \mathbf{K}_{ii}\left(\mathbf{\Xi}_{0}\right)
    $} \\
    \vspace{1.0em}%
    \ForEach{$\omega$ in $\left\{\omega_{1}, ..., \omega_{m}\right\}$}{
        \vspace{1.0em}%
        $\mathbf{\Phi}_{d} \longleftarrow$
        \modeSelector{$
            \mathbf{\Phi}_{n}, \omega
        $} \\
        \vspace{1.0em}%
        \tcp{Generate random samples}
        \For{$i = 1, ..., N_{train}$}{
            \vspace{1.0em}%
            $\mathbf{\Xi}_{i} \longleftarrow$
            \sample{$\mathbf{\Xi}$} \\
            \vspace{1.0em}%
            \tcc{Compute dynamic stiffness matrix $\mathbf{D}\left(\omega, \mathbf{\Xi}_{i}\right)$}
            $\dots$ \\
            \vspace{1.0em}%
            $
                \tilde{\mathbf{\Phi}}_{c} \longleftarrow 
                -\left(
                    \mathbf{D}_{ii}\left(\omega, \mathbf{\Xi}\right)
                \right)^{-1}
                \mathbf{D}_{ib}\left(\omega, \mathbf{\Xi}\right)
            $ \\
            \vspace{1.0em}%
            $
                \mathbf{T} \;\longleftarrow
                \left[
                    \begin{bmatrix}
                        \mathbf{\Phi}_{d} &
                        \tilde{\mathbf{\Phi}}_{c}
                    \end{bmatrix}
                    \begin{bmatrix}
                        \phantom{x}\mathbf{0}\phantom{.} &
                        \mathbf{I}\phantom{x}
                    \end{bmatrix}
                \right]
            $ \\
        \vspace{1.0em}% 
            \tcc{%
                Reduce components with $\mathbf{T}$ and solve for 
                $\mathbf{H} \left(\omega, \mathbf{\Xi}_{j}\right)$
            }
            $\dots$
        }
        \vspace{1.0em}%
        \tcc{Further processes}
        $\dots$
    }
  \caption{The Hybrid UCB Method for FRFs}
\end{algorithm}
\end{center}