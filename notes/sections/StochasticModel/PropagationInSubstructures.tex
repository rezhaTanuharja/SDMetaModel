\documentclass[../DomainDecomposition.tex]{subfiles} 
\bibliography{../../literature}

\begin{document} 

\subsection{Uncertainties Propagation in Substructural Level}

Let the r-th substructure's matrix be functions of a set of random variables $\mathbf{x}$. 
The transformation matrix is given by the following: 

\begin{equation}
    \mathbf{G}_{r} \left(\mathbf{x}\right)
    =
    \begin{bmatrix}
        \bar{\mathbf{\Phi}}_{r}^{N} \left(\mathbf{x}\right) & 
        \bar{\mathbf{\Phi}}_{r}^{C} \left(\mathbf{x}\right) \\
        \mathbf{0} & \mathbf{I} 
    \end{bmatrix}
    \label{Random Transformation Matrix}
\end{equation}

Equation \eqref{Random Transformation Matrix} implies that internal and constraint modes need to be computed for every realisation of the random variables. 
This can be a bottleneck in the training data generation because eigenvectors computation is not cheap.  
To tackle this, an observation is made: when uncertainties result in the following forms 

\begin{equation*}
    \mathbf{M} \left(\mathbf{x}\right) 
    =
    S_{m} \left(\mathbf{x}\right) 
    \cdot 
    \mathbf{M}_{0} 
    \phantom{xx}
    \text{and} 
    \phantom{xx}
    \mathbf{K} \left(\mathbf{x}\right) 
    =
    S_{k} \left(\mathbf{x}\right) 
    \cdot 
    \mathbf{K}_{0} 
\end{equation*}

then the internal and constraint modes of the stochastic system are the same as those of the deterministic system. 
Even if this is not the case, it can be assumed that the random solution may still be approximated well by the deterministic modes.
As an example, uncertainties of density value, elasticity modulus, and load will result in the aforementioned forms.  
Therefore, it is proposed to use the deterministic transformation matrix. 
The modal mass and stiffness matrices are then approximated by the following formulas: 

\begin{equation}
    \bar{\mathbf{M}}_{r} \left(\mathbf{x}\right) 
    =
    \mathbf{G}_{r}^{T} 
    \mathbf{M}_{r} \left(\mathbf{x}\right) 
    \mathbf{G}_{r} 
\end{equation}

\begin{equation}
    \bar{\mathbf{K}}_{r} \left(\mathbf{x}\right) 
    =
    \mathbf{G}_{r}^{T} 
    \mathbf{K}_{r} \left(\mathbf{x}\right) 
    \mathbf{G}_{r} 
\end{equation}

In similar fashion, the transformed load vector is approximated by the following formula: 

\begin{equation}
    \bar{\mathbf{F}}_{r} \left(\mathbf{x}\right) 
    =
    \mathbf{G}_{r}^{T} 
    \mathbf{F}_{r} \left(\mathbf{x}\right) 
\end{equation}

Upon transformation, the solution for the stochastic model is computed using the same procedure as the deterministic model. 

\end{document} 
