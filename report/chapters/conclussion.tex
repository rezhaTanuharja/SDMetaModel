% -----------------------------------------------------------------------------
% Master thesis in the study program computational mechanics
%
% B.Sc. Rezha Adrian Tanuharja - 03751261
% M.Sc. Felix Schneider (supervisor)
%
% chapters/conclussion.tex
% Last edited 03 November 2023
% -----------------------------------------------------------------------------

\chapter{Summary and Conclusion}
\label{ch: conclusion}

This study has developed a novel framework for uncertainty quantification of large and complex structures by combining the well-established CB method with the recently developed surrogate model, the NI-RPCE model.

The first part of the proposed framework consists of the hybrid UCB method: a modified CB method for models with varying input parameters.
The hybrid UCB method avoids computing the components' internal modes for each realization of input parameters.
The transformation of the structure's equation of motion uses invariant reference internal modes instead.
Consequently, one only needs to solve a generalized eigenproblem for each component, thus reducing the computational cost.

The second part of the proposed framework consists of the sparse NI-RPCE model.
A newly developed procedure selectively removes bases from the model, thus leading to a sparse model.
The sparse RPCE model yields smaller errors compared to the original RPCE model when training data sizes are small.
Consequently, a lower number of evaluations of the structures' dynamic response is necessary, thus further reducing the overall computational cost.

The proposed framework combines the hybrid UCB method with the sparse NI-RPCE models.
It uses the former to generate training data at a lower computational cost, trains the latter using the data, and uses the sparse NI-RPCE models to estimate the structure's dynamic responses.
Consequently, the framework offers two advantages: lower computational cost to generate each training data and reduce the required number of training data.
This study has demonstrated the framework using a case study: a plate clamped at both of its ends, in which the performance of the new framework is satisfactory.