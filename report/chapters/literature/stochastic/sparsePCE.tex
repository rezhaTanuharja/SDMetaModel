% -----------------------------------------------------------------------------
% Master thesis in the study program computational mechanics
%
% B.Sc. Rezha Adrian Tanuharja - 03751261
% M.Sc. Felix Schneider (supervisor)
%
% chapters/literature/stochastic/sparsePCE.tex
% Last edited 03 November 2023
% -----------------------------------------------------------------------------

\subsection{Sparse PCE Model}
\label{ssec: sparse PCE}

The computation of FRFs is likely the most expensive step in the method above.
Therefore, the number of samples dictates the overall computational cost.
As the number of basis functions in the approximation increases, so does the required number of samples.
There are several procedures to shrink the number of basis functions.
The following is an example of such procedures and is a simplified version of the one in \cite{blatman2010adaptive}.

The procedure starts with an empty approximation.
It enriches the model with one basis function at a time and checks its accuracy.
If accuracy improves significantly, it keeps the basis function in the model.
Otherwise, it removes the basis functions from the model.
These steps aim to include only the basis functions that significantly improve the model's approximation.
Algorithm \ref{alg: sparse PCE addition} illustrates these steps.

% -----------------------------------------------------------------------------
% Master thesis in the study program computational mechanics
%
% B.Sc. Rezha Adrian Tanuharja - 03751261
% M.Sc. Felix Schneider (supervisor)
%
% algorithm/sparsePCEAddition.tex
% Last edited 03 November 2023
% -----------------------------------------------------------------------------

\begin{center}
\begin{algorithm}[H]
    \label{alg: sparse PCE addition}
    \vspace{1.0em}%
    \tcp{Initialize with an empty set of indices}
    $\mathbf{S} \longleftarrow \left\{\phantom{x}\right\}$ \\
    \vspace{1.0em}%
    \tcp{Compute initial error when all coefficients are zero}
    $\epsilon \longleftarrow$
    \err{$\mathbf{Y}_{test}, \mathbf{A}_{test}, \mathbf{0}$} \\
    \vspace{1.0em}%
    \For{$i = 1, ..., N_{p}$}{
        \vspace{1.0em}%
        $S \longleftarrow S\cup\left\{i\right\}$ \\
        \vspace{1.0em}%
        \tcp{Only compute coefficients with index in $S$}
        $\mathbf{x} \:\phantom{\left[S\right]} \longleftarrow$
        \zeros{$N_{p}$} \\
        $\mathbf{x}\left[S\right] \longleftarrow$
        \lsRegression{$\mathbf{Y}_{train}, \mathbf{A}_{train}\left[\;:,\;S\;\right]$} \\
        \vspace{1.0em}%
        $\epsilon^{+} \longleftarrow$
        \err{$\mathbf{Y}_{test}, \mathbf{A}_{test}, \mathbf{x}$} \\
        \vspace{1.0em}%
        \tcp{Keep $i$ in $S$ if error reduction is significant}
        \uIf{$\epsilon -\epsilon^{+} > \epsilon_{0}$}{
            $\epsilon \longleftarrow \epsilon^{+}$
        }
        \Else{
            $S\setminus \left\{i\right\}$
        }
        \vspace{1.0em}%
    }
    \vspace{1.0em}%
  \caption{Basis Addition Steps for Sparse PCE Model}
\end{algorithm}
\end{center}

Subsequently, the procedure removes one basis function at a time from the approximation and checks the accuracy.
If accuracy declines significantly, it reinstates the basis in the approximation.
These steps aim to remove the basis functions that do not significantly improve the model's approximation.
Algorithm \ref{alg: sparse PCE removal} illustrates these steps.

% -----------------------------------------------------------------------------
% Master thesis in the study program computational mechanics
%
% B.Sc. Rezha Adrian Tanuharja - 03751261
% M.Sc. Felix Schneider (supervisor)
%
% algorithm/sparsePCERemoval.tex
% Last edited 03 November 2023
% -----------------------------------------------------------------------------

\begin{center}
\begin{algorithm}[H]
    \label{alg: sparse PCE removal}
    \vspace{1.0em}%
    \ForEach{$i \in S$}{
        \vspace{1.0em}%
        $S\setminus \left\{i\right\}$ \\
        \vspace{1.0em}%
        \tcp{Only compute coefficients with index in $S$}
        $\mathbf{x} \:\phantom{\left[S\right]} \longleftarrow$
        \zeros{$N_{p}$} \\
        $\mathbf{x}\left[S\right] \longleftarrow$
        \lsRegression{$\mathbf{Y}_{train}, \mathbf{A}_{train}\left[\;:,\;S\;\right]$} \\
        \vspace{1.0em}%
        $\epsilon^{-} \longleftarrow$
        \err{$\mathbf{Y}_{test}, \mathbf{A}_{test}, \mathbf{x}$} \\
        \vspace{1.0em}%
        \tcp{Reinstate $i$ in $S$ if error increase is significant}
        \uIf{$\epsilon^{-} -\epsilon > \epsilon_{0}$}{
            $S \longleftarrow S\cup\left\{i\right\}$
        }
        \Else{
            $\epsilon \longleftarrow \epsilon^{-}$
        }
        \vspace{1.0em}%
    }
    \vspace{1.0em}%
  \caption{Basis Removal Steps for Sparse PCE Model}
\end{algorithm}
\end{center}

After the selective inclusion and exclusion steps above, only some of the basis functions are present in the PCE model.
Hence the term sparse PCE model.